\PassOptionsToPackage{dvipsnames}{xcolor}
\documentclass[twoside,11pt]{entics}
\usepackage{enticsmacro}
\usepackage{graphicx}
\usepackage{cmll}
\usepackage{mathpartir}
\usepackage[all]{xy}
%\usepackage{url}
\sloppy
% The following is enclosed to allow easy detection of differences in
% ascii coding.
% Upper-case    A B C D E F G H I J K L M N O P Q R S T U V W X Y Z
% Lower-case    a b c d e f g h i j k l m n o p q r s t u v w x y z
% Digits        0 1 2 3 4 5 6 7 8 9
% Exclamation   !           Double quote "          Hash (number) #
% Dollar        $           Percent      %          Ampersand     &
% Acute accent  '           Left paren   (          Right paren   )
% Asterisk      *           Plus         +          Comma         ,
% Minus         -           Point        .          Solidus       /
% Colon         :           Semicolon    ;          Less than     <
% Equals        =           Greater than >          Question mark ?
% At            @           Left bracket [          Backslash     \
% Right bracket ]           Circumflex   ^          Underscore    _
% Grave accent  `           Left brace   {          Vertical bar  |
% Right brace   }           Tilde        ~

%%%%%%%%%%%%%%%%%%%%%%%%%%%%%%%%%%%%%%%%%%%%%%%%%%%%%%%%%%%%%%%%%%%%%%%%%%%%%%%%
%% Your Corresponding Editor will provide the following information:          %%
%%%%%%%%%%%%%%%%%%%%%%%%%%%%%%%%%%%%%%%%%%%%%%%%%%%%%%%%%%%%%%%%%%%%%%%%%%%%%%%%

\def\conf{MFPS 2024}
\volume{NN}%
\def\volu{NN}%
\def\papno{nn}%

%%%%%%%%%%%%%%%%%%%%%%%%%%%%%%%%%%%%%%%%%%%%%%%%%%%%%%%%%%%%%%%%%%%%%%%%%%%%%%%%
%% Please fill in the following information:                                  %%
%%%%%%%%%%%%%%%%%%%%%%%%%%%%%%%%%%%%%%%%%%%%%%%%%%%%%%%%%%%%%%%%%%%%%%%%%%%%%%%%
\def\lastname{Atkey, Kokke}
% ^-- If more than three authors, use et al.
\def\runtitle{Semantic Cut Elimation for MAV}
% ^-- Short title appears in the running header on even pages. 
\def\copynames{R. Atkey, W. Kokke}
% ^-- Fill in the first initial and last name of the authors

%%%%%%%%%%%%%%%%%%%%%%%%%%%%%%%%%%%%%%%%%%%%%%%%%%%%%%%%%%%%%%%%%%%%%%%%%%%%%%%%
%% Claiming Creative Commons copyright here:                                  %%
%%%%%%%%%%%%%%%%%%%%%%%%%%%%%%%%%%%%%%%%%%%%%%%%%%%%%%%%%%%%%%%%%%%%%%%%%%%%%%%%
\def\CCB{CC@symbol}
% ^-- Be sure the correct Creative Commons copyright symbol is chosen.
%     See Section 2 of:
%     https://mirrors.concertpass.com/tex-archive/fonts/ccicons/ccicons.pdf

%%%%%%%%%%%%%%%%%%%%%%%%%%%%%%%%%%%%%%%%%%%%%%%%%%%%%%%%%%%%%%%%%%%%%%%%%%%%%%%%
%% Preamble:                                                                  %%
%%%%%%%%%%%%%%%%%%%%%%%%%%%%%%%%%%%%%%%%%%%%%%%%%%%%%%%%%%%%%%%%%%%%%%%%%%%%%%%%

% Include agda.sty
%\usepackage{agda}

% Set up unicode support
\usepackage{iftex}
\ifPDFTeX
  \usepackage{fontenc}
  \usepackage[utf8]{inputenc}
  \usepackage{textcomp} % provide euro and other symbols
  \DeclareUnicodeCharacter{2115}{\ensuremath{\mathbb{N}}}
\else % if luatex or xetex
  \usepackage{unicode-math} % this also loads fontspec
\fi

% Set up TODO macros
\usepackage{todonotes}
% \usepackage{regexpatch}
% \makeatletter
% \xpatchcmd{\@todo}{%
%   \setkeys{todonotes}{#1}}{%
%   \setkeys{todonotes}{inline,#1}}{}{}
% \makeatother
\presetkeys{todonotes}{inline}{}
\newcommand{\wen}[2][]{\todo[%
  author=Wen,%
  color=Thistle,%
  textcolor=Black,#1]{#2}}
\newcommand{\bob}[2][]{\todo[%
  author=Bob,%
  color=Turquoise,%
  textcolor=Black,#1]{#2}}

\begin{document}

%%%%%%%%%%%%%%%%%%%%%%%%%%%%%%%%%%%%%%%%%%%%%%%%%%%%%%%%%%%%%%%%%%%%%%%%%%%%%%%%
%% Frontmatter                                                                %%
%%%%%%%%%%%%%%%%%%%%%%%%%%%%%%%%%%%%%%%%%%%%%%%%%%%%%%%%%%%%%%%%%%%%%%%%%%%%%%%%
\begin{frontmatter}
  \title{A Semantic Proof of Cut Elimination for Multiplicative Additive System Virtual\thanksref{ALL}}
  \thanks[ALL]{General thanks to everyone who should be thanked. FIXME: EPSRC}
  \author{Robert Atkey\thanksref{msp}\thanksref{bobemail}}
  \author{Wen Kokke\thanksref{msp}\thanksref{wenemail}}
  \address[msp]{%
    Mathematically Structured Programming\\
    Computer and Information Sciences\\
    University of Strathclyde\\
    Glasgow, Scotland, UK}
  \thanks[bobemail]%
  {Email: \href{robert.atkey@strath.ac.uk}%
    {\texttt{\normalshape robert.atkey@strath.ac.uk}}}
  % ^-- If all authors share same institution, only list the address once, after the second author. There also is a link from the first author to the co-author's address to show how to list affiliations to more than one institution, when needed. 
  \thanks[wenemail]%
  {Email: \href{wen.kokke@strath.ac.uk}%
    {\texttt{\normalshape wen.kokke@strath.ac.uk}}}
  \begin{abstract}
    To be written...
  \end{abstract}
  \begin{keyword}
    Please list keywords from your paper here, separated by commas.
  \end{keyword}
\end{frontmatter}

\section{Introduction}

\begin{enumerate}
\item It makes sense to add a sequencing connective to MALL. This is
  motivated by coherence spaces (Retore), or by thinking in terms of
  session types.
\item A sequent calculus is not possible (CITE: Tiu, Strassberger),
  but it is possible to define a proof calculus based on deep
  inference. (Also Retore's proof nets, but it turns out they define a
  different logic).
\item To make sure that a proof system makes sense, we need some kind
  of Cut-elimination theorem that ensures that all proofs can be
  placed into a normal form where it is evident that it is not
  possible to derive all results.
\end{enumerate}

MAV is defined by Horne \cite{Horne15:mav}, who gave a syntactic proof
of cut-elimination and the admissibility of several other rules. We
provide an alternative proof of cut-elimination via a semantic
model. This proof avoids some of the intrincate syntactic reasoning in
Horne's proof and is more modular, as evidenced by the extensions we
present in \autoref{sec:extensions}.

Cut-elimination for MALL by construction of a semantic model is due to
Okada \cite{Okada99:psc}, who shows that the a Phase Space model of
MALL (described by Girard \cite{Girard87:ll} and Troelstra
\cite{TroelstraXX:lnll}) can be constructed from cut-proof proofs. The
completeness of this model directly yields the existence of a cut-free
proof for every proof constructible in the MALL sequent calculus. For
MAV, we cannot use Phase Spaces due to the 

The same technique has been used for MALL with fixpoints (CITE:
Saurin) and for Bunched Implications with extensions (CITE: Frumin).


\subsection{Contributions}

\begin{enumerate}
\item We define an algebraic semantics for MAV. Due to the close
  connection between Deep Inference and categorical/algebraic logic
  (CITE: Hughes), this is a relatively straightforward adaptation, but
  is needed for what follows. SECTIONREF.
\item We demonstrate that a useful class of MAV-algebras can be
  constructed from 
\item 
\end{enumerate}

\section{Deep Inference and Multiplicative-Additive System Virtual}

TODO: introduce MAV and Deep Inference.

Formulas:
\begin{displaymath}
  P, Q ::= a \mid a^\perp \mid I \mid P \otimes Q \mid P \parr Q \mid P \with Q \mid P \oplus Q \mid P; Q
\end{displaymath}

Negation:
\begin{displaymath}
  \begin{array}{lcl}
    (a)^\perp           & = & a^\perp \\
    (a^\perp)^\perp     & = & a \\
    (I)^\perp           & = & I \\
    (P \otimes Q)^\perp & = & (P)^\perp \parr   (Q)^\perp \\
    (P \parr   Q)^\perp & = & (P)^\perp \otimes (Q)^\perp \\
    (P \with   Q)^\perp & = & (P)^\perp \oplus  (Q)^\perp \\
    (P \oplus  Q)^\perp & = & (P)^\perp \with   (Q)^\perp \\
    (P ; Q)^\perp       & = & (P)^\perp ; (Q)^\perp
  \end{array}
\end{displaymath}

\section{Semantics of MAV}

\subsection{$*$-autonomous and Duoidal Preorders}

\subsection{MAV Algebras}

\subsection{Interpretation}

\subsection{Chu Construction and Self-Dual Monoids}

\newcommand{\Chu}{\mathrm{Chu}}
\newcommand{\op}{\mathrm{op}}

To construct suitable MAV Algebras, we use the preorder version of the
Chu construction \cite{barr}. The Chu construction builds
$*$-autonomous categories from symmetric monoidal closed categories
with pullbacks. In the preorder case, the requirement for preorders
simplifies to binary meets. Therefore, for this section, we let
$(A, \leq, \bullet, \epsilon, -\kern-9pt\bullet)$ be a symmetric
monoidal closed preorder with all binary meets.

\begin{definition}
  Let $k$ be an element of $A$. $\Chu(A, k)$ is the preorder with
  elements comprised of pairs $(a^+, a^-)$ such that
  $a^+ \bullet a^- \leq k$, with ordering
  $(a^+,a^-) \sqsubseteq (b^+, b^-)$ when $a^+ \leq b^+$ and
  $b^- \leq a^-$.
\end{definition}

\begin{theorem}
  $\Chu(A, k)$ is a $*$-autonomous preorder with symmetric monoidal
  structure and negation defined as:
  \begin{mathpar}
    (a^+, a^-) \otimes (b^+, b^-) = (a^+ \bullet b^+, (b^+ \multimap a^-) \land (a^+ \multimap b^-))

    I = (\epsilon, k)

    (a^+,a^-)^\perp = (a^-, a^+)
  \end{mathpar}
\end{theorem}

\begin{theorem}
  If $A$ has binary joins, then $\Chu(A, k)$ has binary meets, given
  by $(a^+,a^-) \land (b^+,b^-) = (a^+ \land b^+, a^- \lor b^-)$.
\end{theorem}

\paragraph{Self-dual duoidal structure}

These two theorems give us enough structure to interpret the
multiplicative and additive constructs of MAV. We need to also
interpret the self-dual $P; Q$ connective. Let us assume that we have
an operation $-\rhd- : A \times A \to A$ that we want to induce
self-dual operator on $\Chu(A, k)$. By the way the Chu construction is
defined, we can make an operator that is self-dual by definition:
\begin{displaymath}
  (a^+, a^-) ; (b^+, b^-) = (a^+ \rhd b^+, a^- \rhd b^-)
\end{displaymath}
for this to work, we need to assume duoidality.

And that the dualising object $k$ is a
satisfies $k \rhd k \leq k$. If $\rhd$ has a unit $j$, then $(j, j)$ is a unit if


The Chu construction allows us to state
directly that any operator we define on $\Chu(A, k)$ is self dual by
defining as the same

\subsection{Monoids on Lower Sets}

Lower sets: $X : A^\op \to \Omega$.



\subsection{Closure Operators on Lower Sets}

need the ``preorder of elements'' $\int X$ of a lower set $X$.

need trees

\begin{displaymath}
  \mathsf{C}(X) = \{ x \mid t \in \mathrm{Tree}(\int X), x \leq \mathrm{combine}(t) \}
\end{displaymath}

\begin{theorem}
  $\mathsf{C}$ is a closure operator. Moreover, it is lax monoidal wrt
  to one operator, and strong monoidal wrt the other.
\end{theorem}



\section{Semantic Cut-Elimination}

\subsection{Constructing a preorder from MAV}

\subsection{The Okada Property}

\section{Extensions}
\label{sec:extensions}


\bibliographystyle{./entics}
\bibliography{paper}

\end{document}
