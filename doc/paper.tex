\PassOptionsToPackage{dvipsnames}{xcolor}
\documentclass[twoside,11pt]{entics}
\usepackage{enticsmacro}
\usepackage[fixamsmath]{mathtools}
\usepackage{mathcommand}
\usepackage{graphicx}
\usepackage{cmll}
\usepackage{mathpartir}
\usepackage[all]{xy}
\usepackage[inline]{enumitem}
\usepackage{fdsymbol}
\usepackage{stmaryrd}
\usepackage{parskip}
\usepackage[UKenglish,all]{foreign}
\usepackage[goodsyntax]{virginialake}
\vllineartrue%
%\usepackage{url}
\sloppy
% The following is enclosed to allow easy detection of differences in
% ascii coding.
% Upper-case    A B C D E F G H I J K L M N O P Q R S T U V W X Y Z
% Lower-case    a b c d e f g h i j k l m n o p q r s t u v w x y z
% Digits        0 1 2 3 4 5 6 7 8 9
% Exclamation   !           Double quote "          Hash (number) #
% Dollar        $           Percent      %          Ampersand     &
% Acute accent  '           Left paren   (          Right paren   )
% Asterisk      *           Plus         +          Comma         ,
% Minus         -           Point        .          Solidus       /
% Colon         :           Semicolon    ;          Less than     <
% Equals        =           Greater than >          Question mark ?
% At            @           Left bracket [          Backslash     \
% Right bracket ]           Circumflex   ^          Underscore    _
% Grave accent  `           Left brace   {          Vertical bar  |
% Right brace   }           Tilde        ~

%%%%%%%%%%%%%%%%%%%%%%%%%%%%%%%%%%%%%%%%%%%%%%%%%%%%%%%%%%%%%%%%%%%%%%%%%%%%%%%%
%% Your Corresponding Editor will provide the following information:          %%
%%%%%%%%%%%%%%%%%%%%%%%%%%%%%%%%%%%%%%%%%%%%%%%%%%%%%%%%%%%%%%%%%%%%%%%%%%%%%%%%

\def\conf{MFPS 2024}
\volume{NN}%
\def\volu{NN}%
\def\papno{nn}%

%%%%%%%%%%%%%%%%%%%%%%%%%%%%%%%%%%%%%%%%%%%%%%%%%%%%%%%%%%%%%%%%%%%%%%%%%%%%%%%%
%% Please fill in the following information:                                  %%
%%%%%%%%%%%%%%%%%%%%%%%%%%%%%%%%%%%%%%%%%%%%%%%%%%%%%%%%%%%%%%%%%%%%%%%%%%%%%%%%
\def\lastname{Atkey, Kokke}
% ^-- If more than three authors, use et al.
\def\runtitle{Semantic Cut Elimation for MAV}
% ^-- Short title appears in the running header on even pages. 
\def\copynames{R. Atkey, W. Kokke}
% ^-- Fill in the first initial and last name of the authors

%%%%%%%%%%%%%%%%%%%%%%%%%%%%%%%%%%%%%%%%%%%%%%%%%%%%%%%%%%%%%%%%%%%%%%%%%%%%%%%%
%% Claiming Creative Commons copyright here:                                  %%
%%%%%%%%%%%%%%%%%%%%%%%%%%%%%%%%%%%%%%%%%%%%%%%%%%%%%%%%%%%%%%%%%%%%%%%%%%%%%%%%
\def\CCB{CC@symbol}
% ^-- Be sure the correct Creative Commons copyright symbol is chosen.
%     See Section 2 of:
%     https://mirrors.concertpass.com/tex-archive/fonts/ccicons/ccicons.pdf

%%%%%%%%%%%%%%%%%%%%%%%%%%%%%%%%%%%%%%%%%%%%%%%%%%%%%%%%%%%%%%%%%%%%%%%%%%%%%%%%
%% Preamble:                                                                  %%
%%%%%%%%%%%%%%%%%%%%%%%%%%%%%%%%%%%%%%%%%%%%%%%%%%%%%%%%%%%%%%%%%%%%%%%%%%%%%%%%

% Abbreviation for `Chapter'
\def\C{Ch.\ }
% Abbreviation for `Footnote'
\def\Fn{Fn.\ }
% Abbreviation for `Section'
\renewcommand{\S}{Sec.\ }

% Set up unicode support
\usepackage{iftex}
\ifPDFTeX
  \usepackage[T1,T5]{fontenc}
  \usepackage[utf8]{inputenc}
  \usepackage[english]{babel}
  \usepackage{newunicodechar}
\else % if luatex or xetex
  \usepackage{unicode-math} % this also loads fontspec
  \usepackage{newunicodechar}
\fi

% `\IfSingleTF`: Does the element consist of a single token?
\ExplSyntaxOn
\cs_new_eq:NN \IfSingleTF \tl_if_single:nTF
\ExplSyntaxOff

% Set up cleveref with custom labels (via Fabrizio Montesi)
\usepackage[nameinlink]{cleveref}
\makeatletter
\newcommand{\customlabel}[4][0]{%
	\protected@write\@auxout{}{\string\newlabel{#3}{{#4}{\thepage}{#4}{#3}{}}}%
	\protected@write\@auxout{}{\string\newlabel{#3@cref}{{[#2][#1][#1]#4}{\thepage}}}}
\makeatother

% Custom labels for rules
\NewDocumentCommand{\RuleName}{m}{\textsc{#1}}
\NewDocumentCommand{\RuleLabel}{s o m}{%
  \IfValueTF{#2}{%
    \customlabel{Rule}{#2}{\RuleName{#3}}%
    \IfBooleanF{#1}{\hypertarget{#2}{\RuleName{#3}}}
  }{%
    \customlabel{Rule}{rule:#3}{\RuleName{#3}}%
    \IfBooleanF{#1}{\hypertarget{rule:#3}{\RuleName{#3}}}
  }}
\crefformat{Rule}{#2(#1)#3}

% Numbered inline equations
\makeatletter
\newcommand*{\inlineequation}[2][]{%
  \begingroup
    % Put \refstepcounter at the beginning, because
    % package `hyperref' sets the anchor here.
    \refstepcounter{equation}%
    \ifx\\#1\\%
    \else
      \label{#1}%
    \fi
    % prevent line breaks inside equation
    \relpenalty=10000 %
    \binoppenalty=10000 %
    % equation
    \ensuremath{%
      % \displaystyle % larger fractions, ...
      #2%
    }%
    ~\@eqnnum%
  \endgroup
}
\makeatother

%%%%%%%%%%%%%%%%%%%%%%%%%%%%%%%%%%%%%%%%%%%%%%%%%%%%%%%%%%%%%%%%%%%%%%%%%%%%%%%%
%% Syntax:                                                                    %%
%%%%%%%%%%%%%%%%%%%%%%%%%%%%%%%%%%%%%%%%%%%%%%%%%%%%%%%%%%%%%%%%%%%%%%%%%%%%%%%%

% Metavariables
\newcommandPIE\va[0]{\alpha#1#2#3}
\newcommandPIE\vb[0]{\beta#1#2#3}
\newcommandPIE\vP[0]{P#1#2#3}
\newcommandPIE\vQ[0]{Q#1#2#3}
\newcommandPIE\vR[0]{R#1#2#3}
\newcommandPIE\vS[0]{S#1#2#3}
\newcommandPIE\vC[0]{\mathcal{C}#1#2#3}
\newcommandPIE\vD[0]{\mathcal{D}#1#2#3}
\newcommandPIE\vGG[0]{\Gamma#1#2#3}
\newcommandPIE\vGD[0]{\Delta#1#2#3}

% Connectives

% \newcommand{\vDual}[1]{\IfSingleTF{#1}{#1}{(#1)}^{\perp}}
\newcommand{\vDual}[1]{\overline{#1}}
\newcommand{\vPos}[1]{#1^{+}}%{#1}%
\newcommand{\vNeg}[1]{#1^{-}}%{\bar{#1}}%

% NOTE: Unit is defined to match `\vP` in width.
\newlength{\vPWidth}
\settowidth{\vPWidth}{$\vP$}

\def\vUnit{%
  \ensuremath{%
    \mathord{%
      \makebox[\vPWidth][c]{\ensuremath{\mathnormal\circ}}}}}

% NOTE: Binary connectives are defined to be the same width.
\usepackage{tikz}
\newlength{\vlpaWidth}
\settowidth{\vlpaWidth}{$\vlpa$}
\newlength{\vlseWidth}
\settowidth{\vlseWidth}{$\vlse$}
\newlength{\vOpWidth}
\pgfmathsetlength{\vOpWidth}{max(\vlpaWidth,\vlseWidth)}

\def\vTens{%
  \ensuremath{%
    \mathbin{%
      \makebox[\vOpWidth][c]{\ensuremath{\vlte}}}}}
\def\vParr{%
  \ensuremath{%
    \mathbin{%
      \makebox[\vOpWidth][c]{\ensuremath{\vlpa}}}}}
\def\vPlus{%
  \ensuremath{%
    \mathbin{%
      \makebox[\vOpWidth][c]{\ensuremath{\vlor}}}}}
\def\vWith{%
  \ensuremath{%
    \mathbin{%
      \makebox[\vOpWidth][c]{\ensuremath{\vlan}}}}}
\def\vSeq{%
  \ensuremath{%
    \mathbin{%
      \makebox[\vOpWidth][c]{\ensuremath{\vlse}}}}}
\def\vTo{%
  \ensuremath{%
    \mathbin{%
      \makebox[\vOpWidth][c]{\ensuremath{\vlli}}}}}

% Contexts
\newcommand{\vEmpty}[0]{\ensuremath{\vlhole}}
\newcommand{\vPlug}[1]{\ensuremath{\{#1\}}}

% Relations
\NewDocumentCommand{\vEquiv}{s}{%
  \ensuremath{\simeq}}
\NewDocumentCommand{\vInferFrom}{s}{%
  \ensuremath{%
    \mathrel{%
      \longrightarrow\IfBooleanTF{#1}{^\star}{}}}}
\NewDocumentCommand{\vInferTo}{s}{%
  \ensuremath{%
    \mathrel{%
      \longleftarrow\IfBooleanTF{#1}{^\star}{}}}}
\NewDocumentCommand{\vBiInfer}{s}{%
  \ensuremath{%
    \mathrel{%
      \longleftrightarrow\IfBooleanTF{#1}{^\star}{}}}}

%%%%%%%%%%%%%%%%%%%%%%%%%%%%%%%%%%%%%%%%%%%%%%%%%%%%%%%%%%%%%%%%%%%%%%%%%%%%%%%%
%% Semantics:                                                                 %%
%%%%%%%%%%%%%%%%%%%%%%%%%%%%%%%%%%%%%%%%%%%%%%%%%%%%%%%%%%%%%%%%%%%%%%%%%%%%%%%%

% Connectives

\newcommand{\aDual}[1]{\neg\IfSingleTF{#1}{#1}{(#1)}}
% \newcommand{\aDual}[1]{\overline{#1}}

\def\aUnit{%
  \ensuremath{%
    \mathord{%
      \makebox[\vPWidth][c]{\ensuremath{\mathnormal\circ}}}}}
\def\aTens{%
  \ensuremath{%
    \mathbin{%
      \makebox[\vOpWidth][c]{\ensuremath{\vlte}}}}}
\def\aParr{%
  \ensuremath{%
    \mathbin{%
      \makebox[\vOpWidth][c]{\ensuremath{\vlpa}}}}}
\def\aPlus{%
  \ensuremath{%
    \mathbin{%
      \makebox[\vOpWidth][c]{\ensuremath{\vlor}}}}}
\def\aWith{%
  \ensuremath{%
    \mathbin{%
      \makebox[\vOpWidth][c]{\ensuremath{\vlan}}}}}
\def\aSeq{%
  \ensuremath{%
    \mathbin{%
      \makebox[\vOpWidth][c]{\ensuremath{\vlse}}}}}
\def\aTo{%
  \ensuremath{%
    \mathbin{%
      \makebox[\vOpWidth][c]{\ensuremath{\vlli}}}}}

\newcommand{\ChuEmbed}{\eta^c}
\newcommand{\ClosedLowerEmbed}{\eta^+}
\newcommand{\LowerEmbed}{\eta}

%%%%%%%%%%%%%%%%%%%%%%%%%%%%%%%%%%%%%%%%%%%%%%%%%%%%%%%%%%%%%%%%%%%%%%%%%%%%%%%%
%% Agda:                                                                      %%
%%%%%%%%%%%%%%%%%%%%%%%%%%%%%%%%%%%%%%%%%%%%%%%%%%%%%%%%%%%%%%%%%%%%%%%%%%%%%%%%

% Include agda.sty
% \usepackage[bw]{agda}
% \usepackage{ifthen}

% \usepackage{textcomp} % provide euro and other symbols
% \newunicodechar{−}{\ensuremath{\mathnormal-}}
% \newunicodechar{α}{\ensuremath{\alpha}}
% \newunicodechar{⊗}{\vTens}
% \newunicodechar{⅋}{\vParr}
% \newunicodechar{⊕}{\vPlus}
% \newunicodechar{▷}{\vSeq}
% \newunicodechar{∼}{\ensuremath{\mathn ormal\sim}}
% \newunicodechar{≡}{\ensuremath{\mathnormal\equiv}}
% \newunicodechar{⟶}{\ensuremath{\mathnormal\longrightarrow}}
% \newunicodechar{⋆}{\ensuremath{\mathnormal\star}}
% \newunicodechar{⁺}{\ensuremath{^{+}}}
% \newunicodechar{���}{\ensuremath{^{-}}}
% \newunicodechar{¹}{\ensuremath{^{1}}}
% \newunicodechar{′}{\ensuremath{^{\prime}}}
% \newunicodechar{∀}{\ensuremath{\mathnormal\forall}}

% Set up TODO macros
\usepackage{todonotes}
\presetkeys{todonotes}{inline}{}
\newcommand{\wen}[2][]{\todo[%
  author=Wen,%
  color=Thistle,%
  textcolor=Black,#1]{#2}}
\newcommand{\bob}[2][]{\todo[%
  author=Bob,%
  color=Turquoise,%
  textcolor=Black,#1]{#2}}

\begin{document}

%%%%%%%%%%%%%%%%%%%%%%%%%%%%%%%%%%%%%%%%%%%%%%%%%%%%%%%%%%%%%%%%%%%%%%%%%%%%%%%%
%% Frontmatter                                                                %%
%%%%%%%%%%%%%%%%%%%%%%%%%%%%%%%%%%%%%%%%%%%%%%%%%%%%%%%%%%%%%%%%%%%%%%%%%%%%%%%%
\begin{frontmatter}
  \title{Semantic Cut Elimination for Deep Inference\thanksref{ALL}}
  \thanks[ALL]{General thanks to everyone who should be thanked. FIXME: EPSRC}
  \author{Robert Atkey\thanksref{msp}\thanksref{bobemail}}
  \author{Wen Kokke\thanksref{msp}\thanksref{wenemail}}
  \address[msp]{%
    Mathematically Structured Programming\\
    Computer and Information Sciences\\
    University of Strathclyde\\
    Glasgow, Scotland, UK}
  \thanks[bobemail]%
  {Email: \href{robert.atkey@strath.ac.uk}%
    {\texttt{\normalshape robert.atkey@strath.ac.uk}}}
  % ^-- If all authors share same institution, only list the address once, after the second author. There also is a link from the first author to the co-author's address to show how to list affiliations to more than one institution, when needed. 
  \thanks[wenemail]%
  {Email: \href{wen.kokke@strath.ac.uk}%
    {\texttt{\normalshape wen.kokke@strath.ac.uk}}}
  \begin{abstract}
    \emph{Multiplicative-Additive System Virtual} (MAV) is a logic
    that extends \emph{Multiplicative-Additive Linear Logic} with a
    self-dual non-commutative operator expressing the concept of
    ``before'' or ``sequencing''. MAV is also an extenson of the the
    logic \emph{Basic System Virtual} (BV) with additives. Formulas in
    BV have an appealing reading as processes with parallel and
    sequential composition. MAV adds internal and external choice
    operators. They are also closely related to \emph{Concurrent
      Kleene Algebras}.

    \medskip

    Proof systems for MAV and BV
    are \emph{Deep Inference} systems, which allow inference rules to
    be applied anywhere inside a formula.
    As with any proof system, a key question is whether proofs in MAV
    can be reduced to a normal form, removing detours and the
    introduction of formulas not present in the original goal. In
    sequent calcluli systems, this property is referred to as
    \emph{Cut elimination}. Deep Inference systems have an analogous
    Cut rule and other rules that are not present in normalised
    proofs. Cut elimination for Deep Inference systems has the same
    metatheoretic benefits as for Sequent Calculi systems, including
    consistency and decidability.

    \medskip

    Proofs of Cut elimination for BV and MAV present in the literature
    have relied on intrincate syntactic reasoning and complex
    termination measures. For Linear Logic, Okada has pioneered
    semantic proofs of cut elimination, using Girard's Phase Space model of
    Linear Logic and techniques akin to Normalisation by Evaluation in
    $\lambda$-calculus, which avoid this intrincate reasoning, but
    these have not been extended to the Deep Inference systems for BV
    and MAV.

    \medskip

    In this work, we present a concise semantic proof that all MAV
    proofs can be reduced to a normal form avoid the Cut rule and
    other ``non analytic'' rules. Due to the self-dual ``before''
    connective, we cannot use Okada's Phase Space technique, which
    relies on closure under double negation. We build the model more
    directly using closed lower sets and the Chu construction. We also
    develop soundness and completeness proofs of MAV (and BV) with
    respect to a class of models. We have mechanised all our proofs in
    the Agda proof assistant, which provides both assurance of their
    correctness as well as yielding an executable normalisation
    procedure.
  \end{abstract}
  \begin{keyword}
    Please list keywords from your paper here, separated by commas.
  \end{keyword}
\end{frontmatter}

\section{Introduction}\label{sec:introduction}

We present an algebraic semantics and semantic proof of cut elimination for the multiplicative-additive deep inference calculus MAV~\cite{Horne15:mav}, an extension of multiplicative-additive linear logic~\cite[MALL]{Girard87:ll} with a self-dual non-commutative connective.

Deep Inference~\cite{Guglielmi14:di} is generalisation of Gentzen's methodology for designing proof systems which arose from Guglielmi's attempts to relate process algebras~\cite[CCS]{Milner89:CC,Milner80:CCS} to linear logic~\cite{Girard87:ll}.
The problem is that, while the connectives of linear logic capture parallel composition, no connective captures \emph{sequential composition}.
Eventually, Guglielmi's attempts yielded the basic deep inference calculus, BV~\cite{Guglielmi99:bv,Guglielmi07:sis}, which extends multiplicative linear logic~\cite[MLL]{Girard87:ll} with a self-dual non-commutative connective that does capture sequential composition.
Such a connective was already present in Pomset logic~\cite{Retore97:pomset}, where it arose from the study of coherence spaces~\cite[Chapter 8]{GirardTL89:proofs}, which are a semantics for proofs of linear logic.
While BV is similar to Pomset logic, the two are not the same. Recently, Nguyễn and Stra{\ss}burger~\cite{NguyenS22:bvisnotpl} showed that the theorems of BV are a proper subset of the theorems of Pomset logic.

% Neither BV nor Pomset logic has a sequent calculus.
% ~\cite{Tiu06:sisii}

\begin{itemize}
      \item BV~\cite{Guglielmi07:sis} by Guglielmi is an extension of MLL inspired by process algebras such as CCSand Pomset Logic~\cite{Retore97:pomset}.
      \item MAV by Hornes~\cite{Horne15:mav} extends BV with the additives.
      \item BV adds a non-commutative connective to MLL. Such a connective can be motivated by coherence spaces, or by session types.
      \item A shallow sequent calculus cannot capture BV~\cite{Tiu06:sisii}.
      \item To make sure that a proof system makes sense, we need some kind
            of Cut-elimination theorem that ensures that all proofs can be
            placed into a normal form where it is evident that it is not
            possible to derive all results.
\end{itemize}

MAV is defined by Horne \cite{Horne15:mav}, who gave a syntactic proof
of cut-elimination and the admissibility of several other rules. We
provide an alternative proof of cut-elimination via a semantic
model. This proof avoids some of the intrincate syntactic reasoning in
Horne's proof and is more modular, as evidenced by the extensions we
present in \autoref{sec:mav-extensions}.

Cut-elimination for MALL by construction of a semantic model is due to
Okada \cite{Okada99:psc}, who shows that the a Phase Space model of
MALL (described by Girard \cite{Girard87:ll} and Troelstra
\cite{TroelstraXX:lnll}) can be constructed from cut-proof proofs. The
completeness of this model directly yields the existence of a cut-free
proof for every proof constructible in the MALL sequent calculus. For
MAV, we cannot use Phase Spaces due to the

The same technique has been used for MALL with fixpoints (CITE:
Saurin) and for Bunched Implications with extensions (CITE: Frumin).

\subsection{Contribution}\label{sec:contribution}

\begin{enumerate}
  \item We define an algebraic semantics for MAV. Due to the close
        connection between Deep Inference and categorical/algebraic logic
        (CITE: Hughes), this is a relatively straightforward adaptation, but
        is needed for what follows. SECTIONREF.
  \item We demonstrate that a useful class of MAV-algebras can be
        constructed from
  \item
\end{enumerate}

\section{Multiplicative-Additive System Virtual}\label{sec:mav-syntax}

TODO: introduce MAV and Deep Inference.

Formulas:
\begin{displaymath}
  P, Q
  \Coloneq A^{+}
  \mid     A^{-}
  \mid     \Unit
  \mid     P \Tens Q
  \mid     P \Parr Q
  \mid     P \With Q
  \mid     P \Plus Q
  \mid     P \Seq  Q
\end{displaymath}

Negation:
\begin{displaymath}
  \begin{array}{lcl}
    \Dual(\Pos(A))   & = & \Neg(A)                 \\
    \Dual(\Neg(A))   & = & \Pos(A)                 \\
    \Dual(\Unit)     & = & \Unit                   \\
    \Dual(P \Tens Q) & = & \Dual(P) \Parr \Dual(Q) \\
    \Dual(P \Parr Q) & = & \Dual(P) \Tens \Dual(Q) \\
    \Dual(P \With Q) & = & \Dual(P) \Plus \Dual(Q) \\
    \Dual(P \Plus Q) & = & \Dual(P) \With \Dual(Q) \\
    \Dual(P \Seq  Q) & = & \Dual(P) \Seq  \Dual(Q)
  \end{array}
\end{displaymath}

\section{Semantic Models for MAV}\label{sec:mav-semantics}

\newcommand{\LowerSet}[1]{\widehat{#1}}
\newcommand{\Day}[1]{\mathop{\widehat{#1}}}
\newcommand{\ClosedLowerSet}[1]{\widehat{#1}^+}
\newcommand{\ClosedDay}[1]{\mathop{\widehat{#1}^+}}
\newcommand{\Chu}{\mathrm{Chu}}
\newcommand{\op}{\mathrm{op}}

\bob{Short introduction to the section here}

\begin{itemize}
\item We construct algebraic models of MAV
\item MAV-algebras are algebras that MAV is sound for. In some
  sense, this is straightforward, but we identify precisely the
  structure that is needed.
\item The key structure is duoidality, or entropy.
\item We develop some model theory here, and show that MAV-algebras
  can be induced by the weaker structure of MAV-frames. This is
  essential for the cut-free elimination proof.
\end{itemize}

\subsection{Pomonoidal, $*$-autonomous, and Duoidal Structure on Partial Orders}

\begin{definition}
  A \emph{partial order monoid (pomonoid)} $(\bullet, i)$ on a poset
  $(A, \leq)$ comprises a binary operator $\bullet : A \times A \to A$
  that is monotone in both arguments and an element $i \in A$ such
  that the usual monoid laws hold. A \emph{commutative pomonoid} is a
  pomonoid where additionally $x \bullet y = y \bullet x$.
\end{definition}

\begin{definition}
  \bob{residuation}
\end{definition}

The definition of $*$-autonomous category is due to Barr
\cite{Barr_1979}. For our purposes, we need the partial order
analogue, also called a CL algebra by Troelstra
\cite{Troelstra92:lll}.

\begin{definition}
  A \emph{$*$-autonomous partial order} is a structure
  $(A, \leq, \otimes, I, \lnot)$ where $(\otimes, I)$ is a pomonoid on
  $(A, \leq)$ and $\lnot : A^\op \to A$ is an anti-monotone and
  involutive operator on $A$, together satisfying
  $x \otimes y \leq \lnot z$ iff $x \leq \lnot (y \otimes z)$.  A
  $*$-autonomous partial order satisfies \emph{mix} if $\lnot I = I$.
\end{definition}

\begin{remark}
  The structure of a $*$-autonomous partial order has a number of
  immediate consequences, but we leave description of these until
  after the definition of MAV-algebra in Definition
  \ref{defn:mav-algebra}.
\end{remark}

\bob{FIXME: cite that combinatorics book.}

\bob{need to define \emph{entropy} and \emph{entropic} somewhere. I
  think we can have that one pomonoid is \emph{entropic} over another,
  and the structure of having two monoids where one is entropic over
  the other is called duoidal structure.}

\begin{definition}
  Two pomonoids $(\bullet, i)$ and $(\lhd, j)$ on a partial order
  $(A, \leq)$ are in a \emph{duoidal} relationship if the following
  inequalities hold:
  \begin{enumerate}
    \item $(w \lhd x) \bullet (y \lhd z) \leq (w \bullet y) \lhd (x \bullet z)$
    \item $j \bullet j \leq j$
    \item $i \leq i \lhd i$
    \item $i \leq j$
  \end{enumerate}
\end{definition}

\begin{remark}
  In the case when the two pomonoids share a common unit the last
  three conditions for a duoidal relationship are automatically
  satisfied.
\end{remark}

\begin{remark}
  If $\bullet$ is a join, or $\lhd$ is a meet, then all the conditions
  for a duoidal relationship are automatically met for any other
  monoid. \bob{fwd ref to where we use this.}
\end{remark}

\begin{remark}
  The duoidal relationship is not symmetric.
\end{remark}

\subsection{MAV-algebras}

\bob{Change $;$ to $\lhd$}

\begin{definition}\label{defn:mav-algebra}
  An \emph{MAV-algebra} is a structure
  $(A, \leq, \otimes, ;, I, \lnot)$ with the following properties:
  \begin{enumerate}
  \item $(A, \leq, \otimes, I, \lnot)$ is $*$-autonomous and satisfies \emph{mix}.
  \item $(A, \leq, ;, I)$ is a pomonoid.
  \item $;$ is self dual: $\lnot (x ; y) = (\lnot x); (\lnot y)$.
  \item $(\otimes, I)$ is duoidal over $(;, I)$.
  \item $(A, \leq)$ has binary meets, which we write as $x \with y$.
  \end{enumerate}
\end{definition}

\begin{proposition}\label{prop:mav-algebra-consequences}
  Let $(A, \leq, \otimes, ;, I, \lnot)$ be a MAV-algebra.
  \begin{enumerate}
    \item There is another pomonoid structure $(\parr, I)$ on
          $(A, \leq)$, defined as
          $x \parr y = \lnot(\lnot x \otimes \lnot y)$.
    \item $(\otimes, I)$ and $(\parr, I)$ are linearly distributive
          \cite{cockett}:
          $x \otimes (y \parr z) \leq (x \otimes y) \parr z$.
    \item $(A, \leq)$ has binary joins, given by
          $x \oplus y = \lnot (\lnot x \with \lnot y)$.
    \item $\oplus$ distributes over $\otimes$:
          $x \otimes (y \oplus z) = ((x \otimes y) \oplus (x \otimes z)$.
    \item $\with$ distributes over $\parr$:
          $(x \parr z) \with (y \parr z) = (x \with y) \parr z$.
    \item $;$ is entropic over $\parr$:
          $(w \parr x); (y \parr z) \leq (w;y) \parr (x;z)$.
    \item $;$ is entropic over $\with$:
          $(w \with x); (y \with z) \leq (w ; y) \with (x; z)$.
    \item $\oplus$ is entropic over $;$:
          $(w ; x) \oplus (y ; z) \leq (x \oplus y) ; (x \oplus z)$.
  \end{enumerate}
\end{proposition}

\newcommand{\sem}[1]{\llbracket #1 \rrbracket}

\begin{definition}\label{defn:mav-interpretation}
  Let $\mathit{At}$ be a set of atomic propositions. Given an
  MAV-algebra $(A, \leq, \otimes, \lhd, I, \lnot)$ and valuation
  $V : \mathit{At} \to A$, define the interpretation of MAV Formulas
  as follows: $\sem{\vUnit} = I$, $\sem{\alpha} = V(\alpha)$,
  $\sem{\vDual \alpha} = \lnot V(\alpha)$,
  $\sem{\vP\vTens\vQ} = \sem{\vP} \otimes \sem{\vQ}$,
  $\sem{\vP\vParr\vQ} = \sem{\vP} \parr \sem{\vQ}$,
  $\sem{\vP\vSeq\vQ} = \sem{\vP};\sem{\vQ}$,
  $\sem{\vP\vWith\vQ} = \sem{\vP} \with \sem{\vQ}$, and
  $\sem{\vP\vPlus\vQ} = \sem{\vP} \oplus \sem{\vQ}$.
\end{definition}

\begin{lemma}
  For all $\vP$, $\sem{\vDual{\vP}} = \lnot \sem{\vP}$.
\end{lemma}

\begin{theorem}\label{thm:soundness}
  The interpretation defined in Definition
  \ref{defn:mav-interpretation} is sound: for all formulas $P$, if
  $P \longrightarrow^* I$, then $I \leq \sem{P}$.
\end{theorem}

\begin{proof}
  Each of the required inequalities has been established in
  Proposition \ref{prop:mav-algebra-consequences}.
\end{proof}

\subsection{MAV-frames}

To prove completeness of the cut-free fragment of SMAV, we will
construct a particular MAV-algebra from the formulas and cut-free
proofs. Since cut-free proofs do not a priori have all the necessary
structure for an MAV-algebra, we develop a procedure to construct
MAV-algebra from the lighter requirements of an MAV-frame, defined in
this section. MAV-frames have an intuitive reading as a kind of
process algebra, so we use process algebra-like notation. In the
following sections, we show how every MAV-frame induces an
MAV-algebra, and in Section XXX we will show that the MAV-algebra
constructed from the cut-free proof MAV-frame allows us to prove that
all proofs in SMAV have cut-free normal forms.

\begin{definition}
  An \emph{MAV-frame} is a structure
  $(F, \leq, I, \parallel, \lhd, +)$ where $(F, \leq)$ is a partial
  order, $(F, \leq, I, \parallel)$ is a commutative pomonoid,
  $(F, \leq, I, \lhd)$ is a pomonoid, $+$ is a binary monotone
  function on $(F, \leq)$, and these data satisfy the following
  inequalities:
  \begin{enumerate}
    \item $(w \lhd x) \parallel (y \lhd z) \leq (w \parallel y) \lhd (x \parallel z)$
    \item $(x + y) \parallel z \leq (x \parallel z) + (y \parallel z)$
    \item $(w \lhd x) + (y \lhd z) \leq (w + y) \lhd (x + z)$
    \item $I + I \leq I$
  \end{enumerate}
\end{definition}

\begin{remark}
  An MAV-frame is essentially two duoidal relationships and a
  distributivity law.
\end{remark}

\begin{remark}
  MAV-frames have a intuitive reading as a CCS-like process algebra
  (see Milner \cite{milner89} for an introduction to CCS). If we
  assume the existence of a collection of ``action'' elements
  $a \in F$ and their duals $\overline{a} \in F$, satisfying
  $a \parallel \overline{a} \leq I$, then we can read the constructs
  of an MAV-frame as parallel composition, sequential composition, and
  choice. The ordering is interpreted as a reduction relation. An
  interesting avenue for future work would be to discover to what
  extent MAV can be thought of as a logic for processes in this
  process algebra.
\end{remark}

\begin{proposition}\label{prop:cfmav-frame}
  The \emph{cut-free MAV-frame} $\textsc{CFMav}$ is the partial order
  arising as the quotient of the preorder formed from the formulas of
  MAV and $P \leq Q$ if $P \longrightarrow^* Q$. Define
  $P \parallel Q = P \vParr Q$, $P \lhd Q = P \vSeq Q$,
  $P + Q = P \vWith Q$ and $I = \vUnit$. The required (in)equalities
  follow directly from the definition of $\longrightarrow^*$.
\end{proposition}

\begin{remark}
  The construction of the MAV-frame \textsc{CFMav} does not use the
  $\otimes$ and $\oplus$ structure of MAV directly. We will see in
  Section \ref{sec:mav-cut-elimination} that this structure is
  recovered by duality from the other connectives by the constructions
  in the rest of this section. \bob{Which one is ``positive'' again?}
\end{remark}

\subsection{Constructing MAV-algebras from MAV-frames}

We construct MAV-algebras from MAV-frames in a three step process. In
Section \ref{sec:lower-sets}, we use lower sets and the Day
construction to add meets, joins and residuals for pomonoids to a
partial order. This construction creates joins freely, so we restrict
to $+$-closed lower sets in Section \ref{sec:closed-lower-sets}, which
separates the Day construction of pomonoids into two separate
cases. Finally, we create the $*$-autonomous structure using the Chu
construction in Section \ref{sec:chu}. We maintain the duoidal
structure through each construction.

\subsubsection{Lower Sets and Day pomonoids}
\label{sec:lower-sets}

\begin{definition}
  Given a partial order $(A, \leq)$, the set of lower sets
  $\LowerSet{A}$ consists of subsets $F \subseteq A$ that are
  down-closed: $x \in F$ and $y \leq x$ implies $y \in F$. Lower sets
  are ordered by inclusion. Define the embedding
  $\eta : A \to \LowerSet{A}$ as $\eta(x) = \{ y \mid y \leq x \}$.
\end{definition}

\begin{proposition}
  For any $(A, \leq)$, the function $\eta$ is monotone, and
  $(\LowerSet{A}, \subseteq)$ has meets and joins given by
  intersection and union respectively.
%
%  \bob{and $\eta$ preserves any meets that exist}
\end{proposition}

\begin{proposition}\label{prop:day-construction}
  If $(\bullet, i)$ is a pomonoid on $(A, \leq)$, then there is a
  corresponding \emph{Day pomonoid} $(\Day{\bullet}, \Day{i})$ on
  $\LowerSet{A}$ defined as
  $F \Day\bullet G = \{ z \mid z \leq x \bullet y, x \in F, y \in G
  \}$ and $\Day{i} = \eta(i)$. Moreover:
  \begin{enumerate}
  \item $(\Day{\bullet}, \Day{i})$ is left and right residuated: there
    exists a function
    $\Day{\rightblackspoon} : \LowerSet{A} \times \LowerSet{A} \to
    \LowerSet{A}$ such that ... \bob{FIXME}
  \item If $(\bullet, i)$ is a commutative pomonoid, then so is
    $(\Day{\bullet}, \Day{i})$. In this case, the residuals coincide.
  \item The embedding preserves the monoid:
    $\eta(x \bullet y) = \eta(x) \Day\bullet \eta(y)$.
  \end{enumerate}
\end{proposition}

\begin{remark}
  This is the Day monoidal product \cite{day} restricted to the case
  of partial orders.
\end{remark}

\begin{remark}
  When $(A, \leq)$ is an MAV-frame, Proposition
  \ref{prop:day-construction} gives us two pomonoids
  $(\Day{\parallel}, \Day{I})$ and $(\Day{\lhd}, \Day{I})$ on
  $\LowerSet{A}$. Moreover, the next proposition states that the
  duoidal relationship between these monoids is preserved by the Day
  construction:
\end{remark}

\begin{proposition}\label{prop:lower-set-duoidal}
  If $(\bullet, i)$ is duoidal over $(\lhd, j)$ then
  $(\Day{\bullet}, \Day{i})$ is duoidal over $(\Day{\lhd}, \Day{j})$.
\end{proposition}

\subsubsection{$+$-closed Lower Sets}
\label{sec:closed-lower-sets}

In the phase semantics, formulas are interpreted as elements that are
fixed points of a closure operator defined by a double negation with
respect to the monoid on the original frame. This closure operator
generates a partial order of ``facts'' whose meets and joins exactly
correspond to the syntactic ones when the original monoid is derived
from the proofs. As we mentioned in the introduction, the presence of
a self-dual operator in BV means that we cannot use double negation
closure, and we have to proceed more deliberately to preserve
join-like structure in an MAV-frame when building MAV-algebras. We do
this by defining $+$-closed lower sets as those that are closed under
finite $+$-combinations of their elements. This leads to a closure
operator on lower sets that allows us to immediately deduce that
$+$-closed lower sets form a lattice. We also preserve the
Day-pomonoids from lower sets, but in two different ways, depending on
how the original pomonoid interacts with $+$. In Proposition
\ref{prop:closed-monoid-distrib} we handle pomonoids that distribute
over $+$, and in Propositional \ref{prop:closed-monoid-duoidal} we
handle pomonoids that are entopic under $+$. We need these
constructions to lift the $\parallel$ and $\lhd$ pomonoids from
MAV-frames to $+$-closed lower sets. Finally in this section, we show
that duoidal structure on lower sets from Proposition
\ref{prop:lower-set-duoidal} is preserved in $+$-closed lower sets.

For this section, we assume that $(A, \leq)$ is a partial order with a
monotone binary operation $+ : A \times A \to A$ (we do not assume
that $+$ is a join or even a pomonoid.)

\begin{definition}
  A lower set $F \in \LowerSet{A}$ is \emph{$+$-closed} if $x \in F$
  and $y \in F$ imply $x + y \in F$. $+$-closed lower sets are ordered
  by subset inclusion and form a partial order
  $(\ClosedLowerSet{A}, \subseteq)$.
\end{definition}

\begin{proposition}
  Let $U : \ClosedLowerSet{A} \to \LowerSet{A}$ be the ``forgetful''
  function that forgets the $+$-closed property. There is a monotone
  function $\alpha : \LowerSet{A} \to \ClosedLowerSet{A}$ such that
  for all $F \in \ClosedLowerSet{A}$, $\alpha(U F) = F$ and for all
  $F \in \LowerSet{A}$, $F \subseteq U (\alpha F)$.
\end{proposition}

\begin{proof}
  To define $\alpha$, we close lower sets under all
  $+$-combinations. To this end, for $F \in \LowerSet{A}$, define
  $\mathrm{ctxt}(F)$, the set of all $+$-combinations of $F$
  inductively built from constructors
  $\mathsf{leaf} : F \to \mathrm{ctxt}(F)$ and
  $\mathsf{node} : \mathrm{ctxt}(F) \times \mathrm{ctxt}(F) \to
  \mathrm{ctxt}(F)$. We define the \emph{sum} of a context as
  $\mathit{sum}(\mathsf{leaf}~x) = x$ and
  $\mathit{sum}(\mathsf{node}(c,d)) = \mathit{sum}(c) +
  \mathit{sum}(d)$. Now define:
  $\alpha(F) = \{ x \mid c \in \mathrm{ctxt}(F), x \leq
  \mathit{sum}(c) \}$. This is $+$-closed, by taking the
  $\mathsf{node}$ combination of contexts. $\alpha \circ U$ is
  idempotent because $\alpha$ does not introduce any elements to lower
  sets that are already closed. For arbitrary lower sets $F$,
  $F \subseteq U(\alpha F)$ by the $\mathsf{leaf}$ constructor.
\end{proof}

\begin{definition}
  Define the embedding $\eta^+ : A \to \ClosedLowerSet{A}$ as
  $\eta^+(x) = \alpha(\eta(x))$.
\end{definition}

\begin{remark}
  By this proposition, $U \circ \alpha$ is a closure operator on
  $\LowerSet{A}$ \cite{davey-priestley}, and the closed elements are
  those of $\ClosedLowerSet{A}$. The next proposition is standard for
  showing that meets and joins exist on the closed elements for some
  closure operator.
\end{remark}

\begin{proposition}
  $(\ClosedLowerSet{A}, \subseteq)$ has all meets and joins. In the
  binary case, meets are defined by intersection and joins are defined
  by $F \lor G = \alpha (U F \cup U G)$.
\end{proposition}

\begin{proposition}\label{prop:closed-eta-preserve-joins}
  $\eta^+(x + y) \subseteq \eta^+(x) \lor \eta^+(y)$.
\end{proposition}

\begin{remark}
  Proposition \ref{prop:closed-eta-preserve-joins} is the reason for
  requiring $+$-closure. This property will allow us to prove the
  crucial embedding property for all formulas in Section
  \ref{sec:mav-cut-elimination}.

  \bob{Not an equality, I think, because $+$ isn't necessarily a join.}
\end{remark}

\begin{proposition}\label{prop:closed-monoid-distrib}
  For a commutative pomonoid $(\bullet, i)$ on $(A, \leq)$ that
  distributes over $+$
  ($(x + y) \bullet z \leq (x \bullet z) + (y \bullet z)$), then
  $F \ClosedDay{\bullet} G = \alpha(U F \Day{\bullet} U G)$ and
  $\ClosedDay{j} = \alpha(\Day{j})$ define a residuated commutative
  pomonoid on $\ClosedLowerSet{A}$. Moreover,
  $\eta^+(x \bullet y) = \eta^+(x) \ClosedDay{\bullet} \eta^+(y)$.
\end{proposition}

\begin{proof}
  We can define an operation
  $\bullet^c : \mathrm{ctxt}(F) \times \mathrm{ctxt}(G) \to
  \mathrm{ctxt}(F \Day{\bullet} G)$ that ``multiplies'' two trees,
  such that
  $\mathit{sum}(c) \bullet \mathit{sum}(d) \leq \mathit{sum}(c
  \bullet^c d)$, using the distributivity. This allows us to show that
  $\alpha$ preserves the monoid operation:
  $\alpha F \ClosedDay{\bullet} \alpha F = \alpha (F \Day{\bullet}
  G)$. With this, we can show that the monotonicity, associativity,
  unit, and commutativity properties of $\Day{\bullet}$ transfer over
  to $\ClosedDay{\bullet}$. The definition of the residual from lower
  sets is already $+$-closed, by distributivity.
\end{proof}

\begin{proposition}\label{prop:closed-monoid-duoidal}
  For a pomonoid $(\lhd, j)$ on $(A, \leq)$, if this satisfies
  $(w \lhd x) + (y \lhd z) \leq (w + y) \lhd (x + z)$ then the Day
  construction
  $F \Day{\lhd} G = \{ z \mid z \leq x \lhd y, x \in F, y \in G \}$ on
  lower sets is $+$-closed when $F$ and $G$ are. We write
  $F \ClosedDay{\lhd} G$ to indicate when we mean this construction as
  an operation on $+$-closed lower sets. If $j + j \leq j$, then the
  Day unit $\Day{j} = \eta(j)$ is also closed and we write it as
  $\ClosedDay{j} \in \ClosedLowerSet{A}$. Together
  $(\ClosedDay{\lhd}, \ClosedDay{j})$ form a pomonoid on
  $(\ClosedLowerSet{A}, \subseteq)$. Moreover,
  $\eta^+(x \lhd y) = \eta^+(x) \ClosedDay{\lhd} \eta^+(y)$.
\end{proposition}

\begin{proof}
  Since $+$ is entropic over $(\lhd, j)$, the Day monoid $\Day{\lhd}$
  is automatically $+$-closed. The monoid structure directly
  transfers. Similarly, $\eta(j)$ is automatically $+$-closed since
  $j + j \leq j$.
\end{proof}

\begin{remark}
  Generalising the situation for the unit $j$ in Proposition
  \ref{prop:closed-monoid-duoidal}, $\eta(x)$ is closed for any $x$
  such that $x + x \leq x$. Note that if $+$ were a join on
  $(A, \leq)$, then this would automatically be satisfied.
\end{remark}

\begin{remark}
  We have used the same decoration $\ClosedDay{\bullet}$ and
  $\ClosedDay{\lhd}$ for two separate constructions of pomonoids on
  $+$-closed lower sets. We will be careful to distinguish which we
  mean: in our present application, a symmetric operator like
  $\bullet$ will distribute over $+$ and so $\ClosedDay{\bullet}$ will
  be constructed by Proposition \ref{prop:closed-monoid-distrib}; and
  a non-symmetric operator like $\lhd$ will be entropic with respect
  to $+$ and so $\ClosedDay{\lhd}$ will be constructed by Proposition
  \ref{prop:closed-monoid-duoidal}.
\end{remark}

\begin{remark}
  If we have two pomonoids on $(A, \leq)$ that share a unit, then the
  two constructions of units in Propositions
  \ref{prop:closed-monoid-distrib} and
  \ref{prop:closed-monoid-duoidal} will yield the same element of
  $\ClosedLowerSet{A}$.
\end{remark}

\begin{proposition}
  Let $(\bullet, i)$ and $(\lhd, j)$ be pomonoids on $(A, \leq)$ in a
  duoidal relationship, and assume that $(\bullet, i)$ distributes
  over $+$ (as in Proposition \ref{prop:closed-monoid-distrib}) and
  $+$ is entropic over $(\lhd, j)$ (as in Proposition
  \ref{prop:closed-monoid-duoidal}). Then
  $(\ClosedDay{\bullet}, \ClosedDay{i})$ and
  $(\ClosedDay{\lhd}, \ClosedDay{j})$ are in a duoidal relationship on
  $(\ClosedLowerSet{A}, \subseteq)$.
\end{proposition}

\begin{proof}
  The entropic relationship established in Proposition
  \ref{prop:lower-set-duoidal} carries over thanks to the properties
  of $\alpha$ and $U$. The fact that
  $\ClosedDay{i} \subseteq \ClosedDay{j}$ relies on the condition
  $j + j \leq j$ to collapse $+$-contexts of $j$s.
\end{proof}

\subsubsection{Chu Construction}
\label{sec:chu}

To construct suitable MAV-algebras, we use the poset version of the
Chu construction \cite{barr}. The Chu construction builds
$*$-autonomous categories from symmetric monoidal closed categories
with pullbacks. In the partial order case, the requirement for
pullbacks simplifies to binary meets. Therefore, for this section, we
let $(A, \leq, \bullet, i, \rightblackspoon)$ be a partial order with
a residuated pomonoid structure and all binary meets.

\begin{definition}\label{defn:chu}
  Let $k$ be an element of $A$. $\Chu(A, k)$ is the partial order with
  elements pairs $(a^+, a^-)$ such that $a^+ \bullet a^- \leq k$, with
  ordering $(a^+,a^-) \sqsubseteq (b^+, b^-)$ when $a^+ \leq b^+$ and
  $b^- \leq a^-$.
\end{definition}

\begin{proposition}
  $(\Chu(A, k), \sqsubseteq)$ has a $*$-autonomous structure defined
  as:
  \begin{displaymath}
    (a^+, a^-) \otimes (b^+, b^-) = (a^+ \bullet b^+, (b^+ \rightblackspoon a^-) \land (a^+ \rightblackspoon b^-))
    \qquad
    I = (i, k)
    \qquad
    \lnot(a^+,a^-) = (a^-, a^+)
  \end{displaymath}
\end{proposition}

\begin{remark}
  If we choose $k = i$, then $(\Chu(A, i), \sqsubseteq)$ has
  $*$-autonomous structure that satisfies \emph{mix}.
\end{remark}

\begin{proposition}\label{prop:chu-meets}
  If $A$ has binary joins, then $(\Chu(A, k), \sqsubseteq) x$ has
  binary meets, given by
  $(a^+,a^-) \with (b^+,b^-) = (a^+ \land b^+, a^- \lor b^-)$.
\end{proposition}

\begin{remark}
  Since $(\Chu(A, k), \sqsubseteq)$ is a $*$-autonomous partial order,
  then Proposition \ref{prop:chu-meets} also means that $\Chu(A, k)$
  has all binary joins, with
  $(a^+, a^-) \oplus (b^+, b^-) = (a^+ \lor b^+, a^- \land b^-)$.
\end{remark}

We now turn to the self-dual duoidal structure required to interpret
the $\vSeq$ connective. First we transfer pomonoids from $(A, \leq)$
to self-dual pomonoids on $(\Chu(A, k), \sqsubseteq)$ provided they
interact well with $k$:
\begin{proposition}\label{prop:chu-self-dual}
  Let $(\lhd, j)$ be a pomonoid on $(A, \leq)$ such that $(\bullet, i)$ is entropic over 
  $k \lhd k \leq k$ and $j \leq k$, then
  $x ; y = (x^+ \lhd y^+, x^- \lhd y^-)$ and $J = (j, j)$ form a
  self-dual pomonoid on $\Chu(A, k)$.
\end{proposition}

\begin{proof}
  $x;y$ is well-defined because
  $(x^+ \lhd y^+) \bullet (x^- \lhd y^-) \leq (x^+ \bullet x^-) \lhd
  (y^+ \bullet y^-) \leq k \lhd k \leq k$. $J$ is well defined because
  $j \bullet j \leq j \leq k$. The pomonoid laws all transfer directly.
\end{proof}

\begin{remark}
  When $k = j$, the two conditions in the proposition are
  automatically satisfied. Moreover, if $k = i = j$, then not only
  does the $*$-autonomous structure satisfy \emph{mix}, but we also
  have $I =J$.
\end{remark}

Finally, we need to show that if $(\bullet, j)$ is duoidal over
$(\lhd, j)$, then their Chu counterparts are in the same
relationship. Due to the use of residuals in the definition of
$\otimes$, we need the following fact about duoidal residuated
pomonoids:

\begin{lemma}\label{lem:duoidal-residual}
  If $(\bullet, j)$ is duoidal over $(\lhd, j)$ in a partial order
  $(A, \leq)$ and $(\bullet, j)$ has a residual $\rightblackspoon$,
  then
  $(w \rightblackspoon x) \lhd (y \rightblackspoon z) \leq (w \lhd y)
  \rightblackspoon (x \lhd z)$.
\end{lemma}

\begin{remark}
  Lemma \ref{lem:duoidal-residual} is in some sense the
  ``intuitionistic'' version of the duoidal relationship for $\parr$
  arising as the dual of that for $\otimes$ in a $*$-autonomous
  partial order.
\end{remark}

\begin{proposition}
  If $(\bullet, i)$ is entropic over $(\lhd, j)$ on $(A, \leq)$, and
  $(\lhd, j)$ satisfies the conditions of Proposition
  \ref{prop:chu-self-dual}, then $(\otimes, I)$ and $(;, J)$ are in a
  duoidal relationship on $\Chu(A, k)$.
\end{proposition}

\begin{proof}
  For the positive half of the Chu construction, this is a direct
  consequence of the entropy on $(A, \leq)$. For the negative half, we
  use Lemma \ref{lem:duoidal-residual} and the fact that meets are
  always duoidal.
\end{proof}

\subsection{Construction of MAV-algebras}

\begin{theorem}
  If $(F, \leq, \parallel, \lhd, I, +)$ is an MAV-frame, then
  $(\Chu(\ClosedLowerSet{F}, I), \sqsubseteq)$ has the structure of an
  MAV-algebra.
\end{theorem}

\begin{proof}
  Combining all the propositions above, in the special case when some
  of the units collapse.
\end{proof}

\section{Semantic Cut-Elimination and Proof Normalisation}
\label{sec:mav-cut-elimination}

\newcommand{\ChuEmbed}{\eta^c}
\newcommand{\ClosedLowerEmbed}{\eta^+}
\newcommand{\LowerEmbed}{\eta}

Let $\Chu(\ClosedLowerSet{\textsc{CFMav}}, \ClosedDay{I})$ be the
MAV-algebra constructed from the Cut-free proof MAV-frame (Proposition
\ref{prop:cfmav-frame}), where elements are positive/negative pairs of
$+$-closed lower sets of formulas. We define the valuation of atoms as
$V(\alpha) = \ChuEmbed(\ClosedLowerEmbed(\vDual{\alpha}))$. By Theorem
\ref{thm:soundness}, we have an interpretation of SMAV formulas
$\sem{P}$ such that if $P \longrightarrow^* I$, then $I \leq
sem{P}$. We now prove our main proposition about the MAV-algebra
\textsc{CFMav} that will allow us to derive the admissibility of Cut
and all the other symmetric rules of SMAV.

\begin{proposition}\label{prop:embedding-sem}
  $\sem{P} \leq \lnot (\ChuEmbed(\ClosedLowerEmbed(P)))$
\end{proposition}

\begin{proof}
  By the definition of the Chu construction, this statement breaks
  down into two inclusions between pairs of $+$-closed lower sets:
  \begin{enumerate}
  \item $\ClosedLowerEmbed(P) \subseteq \sem{\vP}^-$
  \item $\sem{\vP}^+ \subseteq \ClosedLowerEmbed(P) \rightblackspoon^+ \ClosedDay{I}$
  \end{enumerate}
  We prove the second assuming the first. It suffices to prove that
  $\sem{\vP}^+ \ClosedDay{\bullet} \ClosedLowerEmbed(P) \leq
  \ClosedDay{I}$, which follows from the first part and the property
  of all Chu-elements that
  $\sem{\vP}^+ \ClosedDay{\bullet} \sem{\vP}^- \leq \ClosedDay{I}$.

  We prove the first part by induction on $P$. In the cases when
  $P = \vUnit$ or $P = \vDual{\alpha}$, we already have
  $\sem{P}^- = \ClosedLowerEmbed(P)$. When $P = \alpha$, we have
  $\sem{\alpha}^- = \ClosedLowerEmbed(\vDual{\alpha})
  \rightblackspoon^+ \ClosedDay{I}$. It suffices to prove that
  $\ClosedLowerEmbed(\alpha) \ClosedDay{\bullet}
  \ClosedLowerEmbed(\vDual{\alpha}) \leq \ClosedDay{I}$, which follows
  from the preservation of monoids by $\ClosedLowerEmbed$ and the
  \RuleLabel{Axiom} rule of MAV.

  When $P = Q \vParr R$, $Q \vWith R$, or $Q \vSeq R$, the result
  follows from preservation of the corresponding monoid structure by
  $\ClosedLowerEmbed$. For example,
  $\ClosedLowerEmbed(Q \vParr R) \leq \ClosedLowerEmbed(Q)
  \ClosedDay{\bullet} \ClosedLowerEmbed(R) \leq \sem{Q}^-
  \ClosedDay{\bullet} \sem{R}^- = \sem{Q \vParr R}^-$.

  When $P = Q \vPlus R$, we have
  $\ClosedLowerEmbed(Q \vPlus R) \leq \ClosedLowerEmbed(Q)$ and
  $\ClosedLowerEmbed(Q \vPlus R) \leq \ClosedLowerEmbed(R)$, by the
  \RuleLabel{Left} and \RuleLabel{Right} rules. Therefore,
  $\ClosedLowerEmbed(Q \vPlus R) \leq \ClosedLowerEmbed(Q) \lor
  \ClosedLowerEmbed(R) \leq \sem{Q}^- \lor \sem{R}^- = \sem{Q \vPlus
    R}^-$.

  When $P = Q \vTens R$, we have
  $\sem{Q \vTens R}^- = (\sem{Q}^+ \rightblackspoon \sem{R}^-) \land
  (\sem{R}^+ \rightblackspoon \sem{Q}^-)$. We prove inclusion in the
  left-hand side, the right-hand side is similar. The key property we need to prove is:
  \begin{equation}\label{eq:interaction-leq}
    \ClosedLowerEmbed(Q \vTens R) \ClosedDay{\bullet} (\ClosedLowerEmbed(Q) \rightblackspoon^+ \ClosedDay{I})
    \leq
    \ClosedLowerEmbed(R)
  \end{equation}
  Using the monoidality and monotonicity of $\alpha$, this property is
  implied by the following property in lower sets
  $\LowerSet{\textsc{CFMav}}$:
  \begin{displaymath}
    \LowerEmbed(Q \vTens R) \Day{\bullet} (U(\ClosedLowerEmbed(Q)) \rightblackspoon \Day{I})
    \leq
    \LowerEmbed(R)
  \end{displaymath}
  which follows from the \RuleLabel{Switch} rule of MAV and
  calculation. Using Inequation \ref{eq:interaction-leq}, and property
  (ii) above, we can prove the inequality we need:
  \begin{displaymath}
    \ClosedLowerEmbed(Q \vTens R) \ClosedDay{\bullet} \sem{Q}^+
    \leq
    \ClosedLowerEmbed(Q \vTens R) \ClosedDay{\bullet} (\ClosedLowerEmbed(Q) \rightblackspoon^+ \ClosedDay{I})
    \leq
    \ClosedLowerEmbed(R)
    \leq
    \sem{R}^-
  \end{displaymath}
  Using the residuation property of $\ClosedDay{\bullet}$ we can conclude.
\end{proof}

\begin{theorem}\label{thm:cut-elim}
  If $P \longrightarrow^* I$ in SMAV, then $P \longrightarrow^* I$ in MAV.
\end{theorem}

\begin{proof}
  By Theorem \ref{thm:soundness}, $P \longrightarrow^* I$ in SMAV
  implies $I \leq \sem{P}$. Combined with Proposition
  \ref{prop:embedding-sem}, we have
  $I \leq \lnot \ChuEmbed(\LowerEmbed(P)$. By Definition
  \ref{defn:chu} of the ordering of Chu elements, we have
  $\ClosedLowerEmbed(P) \subseteq \ClosedDay{I}$. Since
  $P \in \ClosedLowerEmbed(P)$, we have $P \in \ClosedDay{I}$, which
  by definition means that $P \longrightarrow^* I$ in MAV.
\end{proof}

\section{Mechanisation in Agda}
\label{sec:mechanisation}

We developed the proof in the previous two sections in the Agda proof
assistant \cite{Agda264}. The source code for this development is
available at this URL:
\begin{center}
  \url{https://github.com/bobatkey/semantic-cut-elimination}
\end{center}

The formalisation uses setoids to represent sets, reusing definitions
from the Agda Standard Library \cite{AgdaStdlib20}.
% Each of the
% definitions and propositions above links to the corresponding part of
% the development.

We did not attempt to formalise the syntactic proof of generalised
cut-elimination directly. We suspect that this would likely be
extremely involved due to the widespread and implicit use of syntactic
equalties when manipulating formulas, as well as the construction of
the relevant termination measures. The semantic constructions are
relatively straightforward to formalise in Agda.

As well as giving confidence in our results, a key benefit of
formalising the development in a proof assistant like Agda is that the
proof normalisation process (\Cref{thm:cut-elim}) is executable. For
example, we can normalise the one step SMAV proof using
\cref{rule:Axiom} of $\vP \vParr \vDual{\vP}$ where
$\vP = (\vUnit \vPlus \vUnit) \vSeq (\vUnit \vWith \vUnit)$ to a 38
step (of which 9 are $\vInferFrom$ steps) MAV proof, presented online
here:
\begin{center}
  \url{https://bobatkey.github.io/semantic-cut-elimination/MAV.Example.html#908}
\end{center}

% \bob{Wen: do you want to write something about the formalisation of
%   the congruence rules in the proof system?}

\section{Conclusions and Future Work}\label{sec:future-work}

We have presented semantic proof of generalised Cut elimination for
the Multiplicative-Additive System Virtual (MAV), reducing Horne's
approximately 41 page proof to 7 and also providing an Agda
mechanisation. Our proof technique also restricts easily to BV (we
simply skip the use of $+$-closed lower sets and use lower sets
directly instead). We believe that adding additive units ($\top$ and
$0$ in Girard's notation) will be straightforward.

This work opens up several paths for future work. The theory developed
here for lifting Day pomonoids to $+$-closed lower sets should also
enable alternative Cut-elimination proofs for other substructural
logics, such as MALL and Bunched Implications (Okada's technique has
already been applied here by Frumin \cite{Frumin22:psc}). We find the
technique of using $+$-closed lower sets rather than more opaque
closure operators more revealing in how the

We also plan to investigate extensions of MAV with exponentials, as in
the System NEL \cite{GuglielmiS11}, and Kleene Star operators which
can be seen as the exponential for the $\vSeq$ connective. Adding a
Kleene Star would tighten the connection with Concurrent Kleene
Algebras we highlighted in \Cref{remark:cka}. More generally, fixpoint
operators following Baelde \cite{Baelde12} and De, Jafarrahmani and
Saurin \cite{De22:psc}. The latters' use of Okada's technique is not
compatible with Agda's logic because it relies on impredicativity to
construct fixpoints with the double negation closure. We believe that
our more direct predicative technique will be able to use Agda's
inductive and coinductive types.

We also plan to extend our semantics of BV and MAV to a categorical
semantics that considers equalities between proofs as well as
provability. Such a semantics ought to be useful for treating MAV as a
session-types style language, as considered by Ciobanu and Horne
\cite{Ciobanu_2016}. The necessary analogue of MAV-algebras has
already been investigated by Blute, Panangaden and Slavnov
\cite{Blute_2010} as BV-categories, which are Aguiar and Mahajan's
2-monoidal, or duoidal, categories \cite{Aguiar_2010} extended with
duality. The key task will be to categorify the constructions in this
paper to show how the categorical analogue of MAV-frames induces
MAV-categories.


\bibliographystyle{entics}
\bibliography{paper}

% \appendix

% \section{Omitted Definitions}\label{sec:omitted-definitions}


\end{document}
