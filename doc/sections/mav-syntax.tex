\section{The system MAV}\label{sec:mav-syntax}

The structures of MAV are formed from positive and negative atoms ($\vPos\va$ and $\vNeg\va$) drawn from some set of atomic propositions, units ($\vUnit$), the non-commutative connective \emph{seq} ($\vSeq$), the multiplicative connectives \emph{tensor} and \emph{par} ($\vTens$ and $\vParr$) and additive connectives \emph{with} and \emph{plus} ($\vWith$ and $\vPlus$).
\begin{displaymath}
  \vP,\vQ,\vR,\vS
  \Coloneq \vPos\va
  \mid     \vNeg\va
  \mid     \vUnit
  \mid     \vP\vSeq \vQ
  \mid     \vP\vTens\vQ
  \mid     \vP\vParr\vQ
  \mid     \vP\vWith\vQ
  \mid     \vP\vPlus\vQ
\end{displaymath}
Duality ($\vDual\vP$) is an involutive function on structures that obeys the De Morgan laws for the multiplicative and additive connectives and preserves the self-dual connective $\vSeq$.
\begin{displaymath}
  \begin{array}{
      lcl @{\hspace{1cm}}
      lcl @{\hspace{1cm}}
      lcl @{\hspace{1cm}}
      lcl}
    \multicolumn{3}{l}{}
    % \vDual{\vPos\va}
    %  & =
    %  & \vNeg\va
     & \vDual{\vUnit}
     & =
     & \vUnit
     & \vDual{\vP\vTens\vQ}
     & =
     & \vDual\vP \vParr \vDual\vQ
     & \vDual{\vP\vParr\vQ}
     & =
     & \vDual\vP \vTens \vDual\vQ
    \\
    \vDual{\vNeg\va}
     & =
     & \vPos\va
     & \vDual{\vP\vSeq \vQ}
     & =
     & \vDual\vP \vSeq  \vDual\vQ
     & \vDual{\vP\vWith\vQ}
     & =
     & \vDual\vP \vPlus \vDual\vQ
     & \vDual{\vP\vPlus\vQ}
     & =
     & \vDual\vP \vWith \vDual\vQ
  \end{array}
\end{displaymath}
Structures are considered equivalent modulo the equality $\vEquiv$, which is the smallest congruence defined by the associativity, commutativity, and identity laws that ensure that $(\vSeq,\vUnit)$ forms a monoid, and $(\vTens,\vUnit)$ and $(\vParr,\vUnit)$ form commutative monoids.
\begin{displaymath}
  \begin{array}{
      l@{\;\vEquiv\;}ll @{\hspace{1cm}}
      l@{\;\vEquiv\;}ll @{\hspace{1cm}}
      l@{\;\vEquiv\;}ll}
    \vP\vSeq\vUnit
     & \vP
     & \RuleLabel*[seq-runit]{\vSeq-Unit\textsuperscript{R}}
     &
    \vUnit\vSeq\vP
     & \vP
     & \RuleLabel*[seq-lunit]{\vSeq-Unit\textsuperscript{L}}
     &
    \vP\vSeq(\vQ\vSeq\vR)
     & (\vP\vSeq\vQ)\vSeq\vP
     & \RuleLabel*[seq-assoc]{\vSeq-Assoc}
    \\
    \vP\vTens\vUnit
     & \vP
     & \RuleLabel*[tens-unit]{\vTens-Unit}
     &
    \vP\vTens\vQ
     & \vQ\vTens\vP
     & \RuleLabel*[tens-comm]{\vTens-Comm}
     &
    \vP\vTens(\vQ\vTens\vR)
     & (\vP\vTens\vQ)\vTens\vP
     & \RuleLabel*[tens-assoc]{\vTens-Assoc}
    \\
    \vP\vParr\vUnit
     & \vP
     & \RuleLabel*[parr-unit]{\vParr-Unit}
     &
    \vP\vParr\vQ
     & \vQ\vParr\vP
     & \RuleLabel*[parr-comm]{\vParr-Comm}
     &
    \vP\vParr(\vQ\vParr\vR)
     & (\vP\vParr\vQ)\vParr\vP
     & \RuleLabel*[parr-assoc]{\vParr-Assoc}
  \end{array}
\end{displaymath}
Structure contexts are one-hole contexts over structures. Plugging ($\vC\vPlug\vP$) replaces the hole in $\vC$ with $\vP$.
\begin{displaymath}
  \vC,\vD
  \Coloneq \vEmpty
  \mid     \vC\vSeq \vQ
  \mid     \vP\vSeq \vD
  \mid     \vC\vTens\vQ
  \mid     \vP\vTens\vD
  \mid     \vC\vParr\vQ
  \mid     \vP\vParr\vD
  \mid     \vC\vWith\vQ
  \mid     \vP\vWith\vD
  \mid     \vC\vPlus\vQ
  \mid     \vP\vPlus\vD
\end{displaymath}
The inference rules of MAV are presented as a \emph{rewriting system} on structures. As this may be surprising to readers unfamiliar with deep inference, let us examine how this presentation relates to the usual presentation of linear logic.
Rule~(\ref{rule:AxiomLL}), shown below, is the axiom rule in the usual one-sided presentation of linear logic.
In the one-sided presentation, the turnstile is vestigial syntax, and can be removed.
In BV, the $\vParr$ connective plays the same role as the comma does in the antecedent of a linear logic sequent, and the $\vUnit$ plays the same role as the empty sequent, which would give us rule~(\ref{rule:AxiomBV-bad}) for BV.
However, BV's inference rules can work arbitrarily deep in the structure. (Hence, \emph{deep} inference.)
Hence, the axiom for BV is rule~(\ref{rule:AxiomBV}), where $\vC$ is a one-hole structure context.
\begin{center}
  $\inlineequation[rule:AxiomLL]{%
      \vlderivation{\vlin{}{}{\vdash\vP,\vDual\vP}{\vlhy{}}}}$
  \qquad
  $\inlineequation[rule:AxiomBV-bad]{%
      \vlderivation{\vlin{}{}{\vP\vParr\vDual\vP}{\vlhy{\vUnit}}}}$
  \qquad
  $\inlineequation[rule:AxiomBV]{%
      \vlderivation{\vlin{}{}{%
          \vC\vPlug{\vP\vParr\vDual\vP}}{%
          \vlhy{\vC\vPlug{\vUnit}}}}}$
  % \qquad
  % $\inlineequation[rule:mav-axiom]{%
  %     \vP\vParr\vDual\vP\vInferFrom\vUnit}$
\end{center}
Rule (\ref{rule:CutLL}) is the cut rule in the usual one-sided presentation of linear logic.
In rule (\ref{rule:CutLL}), as in any branching inference rule, the branching enforces the \emph{disjointness} of the premise derivations.
In BV, disjointness is internalised by the $\vTens$ connective.
Hence, it plays the same role as branching does in sequent calculus.
This would give us rule (\ref{rule:CutBVBad}) for BV.
However, as BV's inference rules can work arbitrarily deep in the structure, the surrounding context of pars is unnecessary---and too restrictive. Hence, the cut for BV is rule (\ref{rule:CutBV}).
\begin{center}
  $\inlineequation[rule:CutLL]{%
      \vlderivation{\vliin{}{}{%
          \vdash\vGG,\vGG',\vGD,\vGD'}{%
          \vlhy{\vdash\vGG,\vP,\vGG'}}{%
          \vlhy{\vdash\vGD,\vDual\vP,\vGD'}}}}$
  \qquad
  $\inlineequation[rule:CutBVBad]{%
      \vlderivation{\vlin{}{}{%
          \vGG\vParr\vGG'\vParr\vGD\vParr\vGD'}{%
          \vlhy{%
            (\vGG\vParr\vP\vParr\vGG')
            \vTens
            (\vGD\vParr\vDual\vP\vParr\vGD')}}}}$
  \qquad
  $\inlineequation[rule:CutBV]{%
      \vlderivation{\vlin{}{}{%
          \vC\vPlug{\vUnit}}{%
          \vlhy{\vC\vPlug{\vP\vTens\vDual\vP}}}}}$
\end{center}
Beautifully, internalising branching makes the symmetry between the axiom and cut plain to see. In BV, to acknowledge this symmetry and the connection with CSS, the axiom and cut rules are referred to as \emph{interaction} and \emph{co-interaction}.

Proof trees are a cumbersome presentation for BV's derivations---they are convenient for trees, but cumbersome in the absence of branching.
\emph{Rewriting systems}, on the other hand, are a well-known and convenient notation for such derivations.
Hence, in MAV, inference rules are presented as rewrite rules.

\emph{Inference}, written $\vInferFrom$, is the smallest relation defined by the following axioms:
% NOTE: These shenanigans are for cross-text table alignment.
\newlength{\vInferFromDefnWidthOfLBL}%
\settowidth{\vInferFromDefnWidthOfLBL}{$\RuleName{AtomAxiom}$}%
\newlength{\vInferFromDefnWidthOfLHS}%
\settowidth{\vInferFromDefnWidthOfLHS}{$(\vP\vSeq\vQ)\vParr(\vR\vSeq\vS)$}%
\newlength{\vInferFromDefnWidthOfRHS}%
\settowidth{\vInferFromDefnWidthOfRHS}{$(\vP\vParr\vR)\vWith(\vQ\vParr\vR)$}%
\begin{displaymath}
  \begin{array}{lcl@{\hspace{0.5cm}}|@{\hspace{0.5cm}}l}
    \vP\vParr\vDual\vP
     & \vInferFrom
     & \vUnit
     & \makebox[\vInferFromDefnWidthOfLBL][l]{$\RuleLabel{Interact}$}
    \\
    (\vP\vTens\vQ)\vParr\vR
     & \vInferFrom
     & \vP\vTens(\vQ\vParr\vR)
     & \RuleLabel{Switch}
    \\
    \vUnit\vWith\vUnit
     & \vInferFrom
     & \vUnit
     & \RuleLabel{Tidy}
    \\
    (\vP\vSeq\vQ)\vParr(\vR\vSeq\vS)
     & \vInferFrom
     & (\vP\vParr\vR)\vSeq(\vQ\vParr\vS)
     & \RuleLabel{Sequence}
    \\
    \vP\vPlus\vQ
     & \vInferFrom
     & \vP
     & \RuleLabel{Left}
    \\
    \vP\vPlus\vQ
     & \vInferFrom
     & \vQ
     & \RuleLabel{Right}
    \\
    (\vP\vWith\vQ)\vParr\vR
     & \vInferFrom
     & (\vP\vParr\vR)\vWith(\vQ\vParr\vR)
     & \RuleLabel{External}
    \\
    (\vP\vSeq\vQ)\vWith(\vR\vSeq\vS)
     & \vInferFrom
     & (\vP\vWith\vR)\vSeq(\vQ\vWith\vS)
     & \RuleLabel{Medial}
    \\
     &
     &
     &
    \\
    \vUnit
     & \vInferFrom
     & \vP\vTens\vDual\vP
     & \RuleLabel{CoInteract}
    \\
    \vUnit
     & \vInferFrom
     & \vUnit \vPlus \vUnit
     & \RuleLabel{CoTidy}
    \\
    (\vP\vTens\vR)\vSeq(\vQ\vTens\vS)
     & \vInferFrom
     & (\vP\vSeq\vQ)\vTens(\vR\vSeq\vS)
     & \RuleLabel{CoSequence}
    \\
    \vP
     & \vInferFrom
     & \vP\vWith\vQ
     & \RuleLabel{CoLeft}
    \\
    \vQ
     & \vInferFrom
     & \vP\vWith\vQ
     & \RuleLabel{CoRight}
    \\
    (\vP \vTens \vR) \vPlus (\vQ \vTens \vR)
     & \vInferFrom
     & (\vP \vPlus \vQ) \vTens \vR
     & \RuleLabel{CoExternal}
    \\
    (\vP \vPlus \vR) \vSeq (\vQ \vPlus \vS)
     & \vInferFrom
     & (\vP \vSeq \vQ) \vPlus (\vR \vSeq \vS)
     & \RuleLabel{CoMedial}
    \\
     &
     &
     &
    \\
    \vC\vPlug\vP\vInferFrom\vC\vPlug\vQ
     & \text{if}
     & \vP\vInferFrom\vQ
     & \RuleLabel{Mono}
    % \\
    % \vP\vInferFrom\vQ
    %  & \text{if}
    %  & \vP\vEquiv\vQ
    %  & \RuleLabel{Equiv}
  \end{array}
\end{displaymath}
If $\vP\vInferFrom\vQ$, we say that $\vP$ can be inferred from $\vQ$, \ie the arrow points from conclusion to premise.

The inference rules come in dual pairs. For every rule $\vP\vInferFrom\vQ$, there is a dual rule $\vDual\vP\vInferFrom\vDual\vQ$.
The exception is \cref{rule:Switch}, which is self-dual, up to commutativity.
The \cref{rule:CoInteract}, \cref{rule:CoLeft}, and \cref{rule:CoRight} rules introduce of new structures going left-to-right.
Normal proofs, which we define below, avoid these \emph{synthetic} rules.

\begin{definition}\label[defn]{defn:derivation}
  \emph{Derivation}, written $\vInferFrom*$ is the reflexive, transitive closure of inference.  \emph{Invertible inference}, written $\vBiInfer$, is the symmetric core of $\vInferFrom$, \ie ${\vBiInfer} = {\vInferFrom\cap\vInferTo}$. \emph{Invertible derivation}, written $\vBiInfer*$, is the symmetric core of derivation.  \emph{Proofs} are derivations that terminate in the unit, \eg a derivation $\vP\vInferFrom*\vUnit$ is a proof of $\vP$.
\end{definition}

\begin{remark}
  In BV, the structural connectives are usually presented as lists, distinguished only by their brackets: $\vP\vTens\vQ$ is written as $\vls(\vP;\vQ)$; $\vP\vParr\vQ$ is written as $\vls[\vP;\vQ]$; and $\vP\vSeq\vQ$ is written as $\vls<\vP;\vQ>$.
  % The equations defining duality are usually included in the decidable equality on structures.
  Inferences, derivations, and proofs are presented vertically, as (\ref{notation:vlin}), (\ref{notation:vlde}), and (\ref{notation:vlpr}), respectively.
  \begin{center}
    $\inlineequation[notation:vlin]{%
        \vlderivation{\vlin{}{}{\vP}{\vlhy{\vQ}}}}$
    \qquad
    $\inlineequation[notation:vlde]{%
        \vlderivation{\vlde{}{}{\vP}{\vlhy{\vQ}}}}$
    \qquad
    $\inlineequation[notation:vlpr]{%
        \vlderivation{\vlpr{}{}{\vP}}}$
  \end{center}
  The relation between the deductive system for BV and rewrite systems is well-known, \eg by Kahramanogullari~\cite{Kahramanogullari08:maude}, who implements proof search for several deep inference systems in Maude~\cite{ClavelDELMMQ02:maude}.
  Inferences rules are usually named with the combination of a letter and an up or down arrow, \eg \cref{rule:Interact} and \cref{rule:CoInteract} are $\iD$ and $\iU$, respectively. The exception are self-dual rules, which are named with a single letter. For instance, \cref{rule:Switch} is usually named $\sw$.
\end{remark}

\begin{definition}\label{defn:normal}
  A derivation is \emph{normal} when it does not use \cref{rule:CoInteract} nor any of the other (\RuleName{CoX}) rules, and its uses of \cref{rule:Interact} are restricted to \cref{rule:AtomInteract}, as defined by the following axiom:
  \begin{displaymath}
    \begin{array}{lcl@{\hspace{0.5cm}}|@{\hspace{0.5cm}}l}
      \makebox[\vInferFromDefnWidthOfLHS][l]{$\vNeg\va\vParr\vPos\va$}
       & \vInferFrom
       & \makebox[\vInferFromDefnWidthOfRHS][l]{$\vUnit$}
       & \RuleLabel{AtomInteract}
    \end{array}
  \end{displaymath}
\end{definition}
Normal derivations avoid the use of rules that introduce new structures into proofs, and so can be termed \emph{analytic} in contrast to the need for the synthetic rules to synthesise new structures.

Our main result, \Cref{thm:cut-elim}, is that every structure that is provable in full MAV also has a normal proof. Therefore, the system with only analytic rules is complete for MAV provability. Horne \cite{Horne15:mav} proves this result via a syntactic proof involving rewriting and termination measures. In the following sections, we construct a semantic proof that normal proofs are complete for MAV.
