\section{The system MAV}\label{sec:mav-syntax}

The structures of MAV are formed from positive and negative atoms ($\vPos\vA$ and $\vNeg\vA$), units ($\vUnit$), the non-commutative connective \emph{seq} ($\vSeq$), the multiplicative connectives \emph{tensor} and \emph{par} ($\vTens$ and $\vParr$) and additive connectives \emph{with} and \emph{plus} ($\vWith$ and $\vPlus$).
\begin{displaymath}
  \vP,\vQ,\vR,\vS
  \Coloneq \vPos\vA
  \mid     \vNeg\vA
  \mid     \vUnit
  \mid     \vP\vSeq \vQ
  \mid     \vP\vTens\vQ
  \mid     \vP\vParr\vQ
  \mid     \vP\vWith\vQ
  \mid     \vP\vPlus\vQ
\end{displaymath}
Duality ($\vDual\vP$) is an involutive function on structures that obeys the De Morgan laws for the multiplicative and additive connectives and preserves the self-dual connective seq.
\begin{displaymath}
  \begin{array}{
      l@{\;=\;}l @{\hspace{1cm}}
      l@{\;=\;}l @{\hspace{1cm}}
      l@{\;=\;}l @{\hspace{1cm}}
      l@{\;=\;}l}
    \vDual{\vPos\vA}     & \vNeg\vA
                         &
    \vDual{\vUnit}       & \vUnit
                         &
    \vDual{\vP\vTens\vQ} & \vDual\vP \vParr \vDual\vQ
                         &
    \vDual{\vP\vParr\vQ} & \vDual\vP \vTens \vDual\vQ
    \\
    \vDual{\vNeg\vA}     & \vPos\vA
                         &
    \vDual{\vP\vSeq \vQ} & \vDual\vP \vSeq  \vDual\vQ
                         &
    \vDual{\vP\vWith\vQ} & \vDual\vP \vPlus \vDual\vQ
                         &
    \vDual{\vP\vPlus\vQ} & \vDual\vP \vWith \vDual\vQ
  \end{array}
\end{displaymath}

Equivalences:
\begin{displaymath}
  \begin{array}{
      l@{\;\vEquiv\;}ll @{\hspace{1cm}}
      l@{\;\vEquiv\;}ll @{\hspace{1cm}}
      l@{\;\vEquiv\;}ll}
    \vP\vSeq\vUnit
     & \vP
     & \RuleLabel*[seq-runit]{\vSeq-Unit\textsuperscript{R}}
     &
    \vUnit\vSeq\vP
     & \vP
     & \RuleLabel*[seq-lunit]{\vSeq-Unit\textsuperscript{L}}
     &
    \vP\vSeq(\vQ\vSeq\vR)
     & (\vP\vSeq\vQ)\vSeq\vP
     & \RuleLabel*[seq-assoc]{\vSeq-Assoc}
    \\
    \vP\vTens\vUnit
     & \vP
     & \RuleLabel*[tens-unit]{\vTens-Unit}
     &
    \vP\vTens\vQ
     & \vQ\vTens\vP
     & \RuleLabel*[tens-comm]{\vTens-Comm}
     &
    \vP\vTens(\vQ\vTens\vR)
     & (\vP\vTens\vQ)\vTens\vP
     & \RuleLabel*[tens-assoc]{\vTens-Assoc}
    \\
    \vP\vParr\vUnit
     & \vP
     & \RuleLabel*[parr-unit]{\vParr-Unit}
     &
    \vP\vParr\vQ
     & \vQ\vParr\vP
     & \RuleLabel*[parr-comm]{\vParr-Comm}
     &
    \vP\vParr(\vQ\vParr\vR)
     & (\vP\vParr\vQ)\vParr\vP
     & \RuleLabel*[parr-assoc]{\vParr-Assoc}
  \end{array}
\end{displaymath}


\begin{itemize}
  \item Structural equivalences are explicit.
  \item Inference is a congruence.
  \item Inference rules are unary.
  \item Relate to usual presentation of MALL.
  \item Relate to usual presentation of BV.
\end{itemize}

Inference rules:
\begin{displaymath}
  \begin{array}{l@{\;\vInfer\;}l@{\hspace{0.5cm}}|@{\hspace{0.5cm}}l}
    \vP\vParr\vDual\vP
     & \vUnit
     & \RuleLabel{axiom}
    \\
    \vUnit
     & \vP\vTens\vDual\vP
     & \RuleLabel{cut}
    \\
    \vUnit\vWith\vUnit
     & \vUnit
     & \RuleLabel{tidy}
    \\
    (\vP\vTens\vQ)\vParr\vR
     & \vP\vTens(\vQ\vParr\vR)
     & \RuleLabel{switch}
    \\
    (\vP\vSeq\vQ)\vParr(\vR\vSeq\vS)
     & (\vP\vParr\vR)\vSeq(\vQ\vParr\vS)
     & \RuleLabel{sequence}
    \\
    \vP\vPlus\vQ
     & \vP
     & \RuleLabel{left}
    \\
    \vP\vPlus\vQ
     & \vQ
     & \RuleLabel{right}
    \\
    (\vP\vWith\vQ)\vParr\vR
     & (\vP\vParr\vR)\vWith(\vQ\vParr\vR)
     & \RuleLabel{external}
    \\
    (\vP\vSeq\vQ)\vWith(\vR\vSeq\vS)
     & (\vP\vWith\vR)\vSeq(\vQ\vWith\vS)
     & \RuleLabel{medial}
  \end{array}
\end{displaymath}

We write $\vEquiv$ for invertible inference rules, \ie $\vP\vEquiv\vQ$ means $\vP\vInfer\vQ$ and $\vQ\vInfer\vP$.

We write $\vInfer*$ for the reflexive, transitive closure of $\vInfer$.

\begin{remark}
  The usual presentation of BV writes the structure connectives as lists, distinguished only by their brackets: $\vP\vTens\vQ$ is written as $\vls(\vP;\vQ)$; $\vP\vParr\vQ$ is written as $\vls[\vP;\vQ]$; and $\vP\vSeq\vQ$ is written as $\vls<\vP;\vQ>$.
\end{remark}