\section{The system MAV}\label{sec:mav-syntax}

The structures of MAV are formed from positive and negative atoms ($\vPos\va$ and $\vNeg\va$), units ($\vUnit$), the non-commutative connective \emph{seq} ($\vSeq$), the multiplicative connectives \emph{tensor} and \emph{par} ($\vTens$ and $\vParr$) and additive connectives \emph{with} and \emph{plus} ($\vWith$ and $\vPlus$).
\begin{displaymath}
  \vP,\vQ,\vR,\vS
  \Coloneq \vPos\va
  \mid     \vNeg\va
  \mid     \vUnit
  \mid     \vP\vSeq \vQ
  \mid     \vP\vTens\vQ
  \mid     \vP\vParr\vQ
  \mid     \vP\vWith\vQ
  \mid     \vP\vPlus\vQ
\end{displaymath}
Duality ($\vDual\vP$) is an involutive function on structures that obeys the De Morgan laws for the multiplicative and additive connectives and preserves the self-dual connective seq.
\begin{displaymath}
  \begin{array}{
      l@{\;=\;}l @{\hspace{1cm}}
      l@{\;=\;}l @{\hspace{1cm}}
      l@{\;=\;}l @{\hspace{1cm}}
      l@{\;=\;}l}
    \vDual{\vPos\va}     & \vNeg\va
                         &
    \vDual{\vUnit}       & \vUnit
                         &
    \vDual{\vP\vTens\vQ} & \vDual\vP \vParr \vDual\vQ
                         &
    \vDual{\vP\vParr\vQ} & \vDual\vP \vTens \vDual\vQ
    \\
    \vDual{\vNeg\va}     & \vPos\va
                         &
    \vDual{\vP\vSeq \vQ} & \vDual\vP \vSeq  \vDual\vQ
                         &
    \vDual{\vP\vWith\vQ} & \vDual\vP \vPlus \vDual\vQ
                         &
    \vDual{\vP\vPlus\vQ} & \vDual\vP \vWith \vDual\vQ
  \end{array}
\end{displaymath}
Structures are considered equivalent modulo the decidable equality $\vEquiv$, which is the smallest congruence defined by the associativity, commutativity, and identity laws that ensure that $(\vSeq,\vUnit)$ forms a monoid, and $(\vTens,\vUnit)$ and $(\vParr,\vUnit)$ form commutative monoids.
\begin{displaymath}
  \begin{array}{
      l@{\;\vEquiv\;}ll @{\hspace{1cm}}
      l@{\;\vEquiv\;}ll @{\hspace{1cm}}
      l@{\;\vEquiv\;}ll}
    \vP\vSeq\vUnit
     & \vP
     & \RuleLabel*[seq-runit]{\vSeq-Unit\textsuperscript{R}}
     &
    \vUnit\vSeq\vP
     & \vP
     & \RuleLabel*[seq-lunit]{\vSeq-Unit\textsuperscript{L}}
     &
    \vP\vSeq(\vQ\vSeq\vR)
     & (\vP\vSeq\vQ)\vSeq\vP
     & \RuleLabel*[seq-assoc]{\vSeq-Assoc}
    \\
    \vP\vTens\vUnit
     & \vP
     & \RuleLabel*[tens-unit]{\vTens-Unit}
     &
    \vP\vTens\vQ
     & \vQ\vTens\vP
     & \RuleLabel*[tens-comm]{\vTens-Comm}
     &
    \vP\vTens(\vQ\vTens\vR)
     & (\vP\vTens\vQ)\vTens\vP
     & \RuleLabel*[tens-assoc]{\vTens-Assoc}
    \\
    \vP\vParr\vUnit
     & \vP
     & \RuleLabel*[parr-unit]{\vParr-Unit}
     &
    \vP\vParr\vQ
     & \vQ\vParr\vP
     & \RuleLabel*[parr-comm]{\vParr-Comm}
     &
    \vP\vParr(\vQ\vParr\vR)
     & (\vP\vParr\vQ)\vParr\vP
     & \RuleLabel*[parr-assoc]{\vParr-Assoc}
  \end{array}
\end{displaymath}
The inference rules of MAV are presented as a \emph{rewriting system} on structures. As this may be surprising to readers unfamiliar with deep inference, let us examine how this presentation relates to the usual presentation of linear logic.
Rule~(\ref{rule:ll-axiom}) is the axiom rule in the usual one-sided presentation of linear logic.
In the one-sided presentation, the turnstile is vestigial syntax, and can be removed.
In BV, the $\vParr$ connective plays the same role as the comma does in the antecedent of a linear logic sequent, and the $\vUnit$ plays the same role as the empty sequent, which would give us rule~(\ref{rule:bv-axiom-bad}) for BV.
However, BV's inference rules can work arbitrarily deep in the structure. (Hence, \emph{deep} inference.)
Therefore, the axiom for BV is actually rule~(\ref{rule:bv-axiom-good}), where $\vC$ is a one-hole structure context.
In BV, all inference rules are unary.
A branching inference rule enforces \emph{disjointness} of its premise derivations.
In BV, disjointness is internalised by the $\vTens$ connective.
\begin{center}
  $\inlineequation[rule:ll-axiom]{%
      \vlderivation{\vlin{}{}{\vdash\vP,\vDual\vP}{\vlhy{}}}}$
  \qquad
  $\inlineequation[rule:bv-axiom-bad]{%
      \vlderivation{\vlin{}{}{\vP\vParr\vDual\vP}{\vlhy{\vUnit}}}}$
  \qquad
  $\inlineequation[rule:bv-axiom-good]{%
      \vlderivation{\vlin{}{}{%
          \vC\vPlug{\vP\vParr\vDual\vP}}{%
          \vlhy{\vC\vPlug{\vUnit}}}}}$
  \qquad
  $\inlineequation[rule:mav-axiom]{%
      \vP\vParr\vDual\vP\vInferFrom\vUnit}$
  \\[1\baselineskip]
  $\inlineequation[rule:ll-cut]{%
      \vlderivation{\vliin{}{}{%
          \vdash\vGG,\vGG',\vGD,\vGD'}{%
          \vlhy{\vdash\vGG,\vP,\vGG'}}{%
          \vlhy{\vdash\vGD,\vDual\vP,\vGD'}}}}$
  \qquad
  $\inlineequation[rule:ll-cut]{%
      \vlderivation{\vliin{}{}{%
          \vGG,\vGG',\vUnit,\vGD,\vGD'}{%
          \vlhy{\vdash\vGG,\vP,\vGG'}}{%
          \vlhy{\vdash\vGD,\vDual\vP,\vGD'}}}}$
  \qquad
  $\inlineequation[rule:bv-cut-good]{%
      \vlderivation{\vlin{}{}{%
          \vC\vPlug{\vP\vParr\vDual\vP}}{%
          \vlhy{\vC\vPlug{\vUnit}}}}}$
  \qquad
  $\inlineequation[rule:mav-cut]{%
      \vP\vParr\vDual\vP\vInferFrom\vUnit}$
\end{center}
The inference rules of MAV are presented as a \emph{rewrite rules} on structures.
We write $\vP\vInferFrom\vQ$ when $\vP$ can be inferred from $\vQ$.
The axiom of linear logic is typically given as \ref{rule:axiom-ll}.
In MAV, the structure connective $\vParr$ plays the same role as the comma does in the antecedent of a linear logic sequent.
\begin{displaymath}
  \begin{array}{lcl@{\hspace{0.5cm}}|@{\hspace{0.5cm}}l}
    \vP\vParr\vDual\vP
     & \vInferFrom
     & \vUnit
     & \RuleLabel{axiom}
    \\
    \vUnit
     & \vInferFrom
     & \vP\vTens\vDual\vP
     & \RuleLabel{cut}
    \\
    \vUnit\vWith\vUnit
     & \vInferFrom
     & \vUnit
     & \RuleLabel{tidy}
    \\
    (\vP\vTens\vQ)\vParr\vR
     & \vInferFrom
     & \vP\vTens(\vQ\vParr\vR)
     & \RuleLabel{switch}
    \\
    (\vP\vSeq\vQ)\vParr(\vR\vSeq\vS)
     & \vInferFrom
     & (\vP\vParr\vR)\vSeq(\vQ\vParr\vS)
     & \RuleLabel{sequence}
    \\
    \vP\vPlus\vQ
     & \vInferFrom
     & \vP
     & \RuleLabel{left}
    \\
    \vP\vPlus\vQ
     & \vInferFrom
     & \vQ
     & \RuleLabel{right}
    \\
    (\vP\vWith\vQ)\vParr\vR
     & \vInferFrom
     & (\vP\vParr\vR)\vWith(\vQ\vParr\vR)
     & \RuleLabel{external}
    \\
    (\vP\vSeq\vQ)\vWith(\vR\vSeq\vS)
     & \vInferFrom
     & (\vP\vWith\vR)\vSeq(\vQ\vWith\vS)
     & \RuleLabel{medial}
    \\
    \vC\vPlug\vP\vInferFrom\vC\vPlug\vQ
     & \text{if}
     & \vP\vInferFrom\vQ
     & \RuleLabel{mono}
    % \\
    % \vP\vInferFrom\vQ
    %  & \text{if}
    %  & \vP\vEquiv\vQ
    %  & \RuleLabel{equiv}
  \end{array}
\end{displaymath}

We write $\vInferFrom*$ for the reflexive, transitive closure of $\vInferFrom$.
We write $\vBiInfer$ for the symmetric core of $\vInferFrom$ (\ie ${\vBiInfer} = {\vInferFrom\cap\vInferTo}$) and $\vBiInfer*$ for the reflexive, transitive closure of $\vBiInfer$.

\begin{remark}
  The structural connectives of BV are usually presented as lists, distinguished only by their brackets: $\vP\vTens\vQ$ is written as $\vls(\vP;\vQ)$; $\vP\vParr\vQ$ is written as $\vls[\vP;\vQ]$; and $\vP\vSeq\vQ$ is written as $\vls<\vP;\vQ>$.
\end{remark}

A \emph{normal} derivation is a derivation where \cref{rule:cut} is not used and \cref{rule:axiom} is restricted to atoms.

\begin{displaymath}
  \begin{array}{l@{\;\vInferFrom\;}l@{\hspace{0.5cm}}|@{\hspace{0.5cm}}l}
    \vNeg\va\vParr\vPos\va
     & \vUnit
     & \RuleLabel{Axiom}
  \end{array}
\end{displaymath}

\begin{itemize}
  \item Structural equivalences are explicit.
  \item Inference is a congruence.
  \item Inference rules are unary.
  \item Relate to usual presentation of MALL.
  \item Relate to usual presentation of BV.
\end{itemize}
