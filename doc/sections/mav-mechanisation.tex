\section{Mechanisation in Agda}
\label{sec:mechanisation}
We formalised the proofs in the paper in Agda~\cite{Agda264}.
The source code is available at the following URL:
\begin{center}
  \url{https://github.com/bobatkey/semantic-cut-elimination}
\end{center}
Furhtermore, a hyperlinked HTML rendition of the source can be browsed at the following URL:
\begin{center}
  \url{\AgdaBaseUrl}
\end{center}
In \Cref{sec:table-of-statments}, we provide a guide to the mechanisation relating the Definitions, Propositions, and Theorems in the previous sections to the Agda definitions in the mechanisation.
The formalisation uses setoids to represent sets, and reuses definitions from the Agda Standard Library~\cite{AgdaStdlib20} where appropriate.

We did not attempt to formalise Horne's syntactic proof of generalised cut-elimination directly. We suspect that this would likely be quite involved, due to the widespread and implicit use of syntactic equalities when manipulating structures, as well as the construction of the relevant termination measures. The semantic constructions are relatively straightforward to formalise in Agda.

In addition to increasing the confidence in our results, a key benefit of the formalised proof in a proof assistant such as Agda is that the proof normalisation procedure defined by \Cref{thm:cut-elim} is executable.
As an example, we have normalised the one-step proof below:
\begin{displaymath}
  ((\vUnit \vPlus \vUnit) \vSeq (\vUnit \vWith \vUnit))
  \vParr
  ((\vUnit \vWith \vUnit) \vSeq (\vUnit \vPlus \vUnit))
  \xlongrightarrow{\cref{rule:Interact}}
  \vUnit
\end{displaymath}
The proof normalises to a 38-step normal proof, of which 9 are inference steps, and the remainder are (sometimes spurious) equalities.
The example can be found at the following URL:
\begin{center}
  \url{\AgdaBaseUrl/MAV.Example.html}
\end{center}
