\section{Semantics}\label{sec:mav-semantics}

\subsection{$*$-autonomous and Duoidal Partial Orders}

\begin{definition}
  A \emph{partial order monoid (pomonoid)} $(\bullet, i)$ on a poset
  $(A, \leq)$ comprises a binary operator $\bullet : A \times A \to A$
  that is monotone in both arguments and an element $i \in A$ such
  that the usual monoid laws hold. A \emph{commutative pomonoid} is a
  pomonoid where additionally $x \bullet y = y \bullet x$.
\end{definition}

\begin{definition}
  A \emph{$*$-autonomous partial order} is ... TODO
\end{definition}

\begin{remark}
  The structure of a $*$-autonomous partial order has a number of
  immediate consequences, but we leave description of these until
  after the definition of MAV algebra below. \bob{because there are
    even more consequences in the presence of additive and duoidal
    structure}
\end{remark}

FIXME: cite that combinatorics book.

\bob{need to define \emph{entropy} and \emph{entropic} somewhere}

\begin{definition}
  Two pomonoids $(\bullet, i)$ and $(\rhd, j)$ on a partial order
  $(A, \leq)$ are in a \emph{duoidal} relationship if the following
  inequalities hold:
  \begin{enumerate}
  \item $(w \rhd x) \bullet (y \rhd z) \leq (w \bullet y) \rhd (x \bullet z)$
  \item $j \bullet j \leq j$
  \item $i \leq i \rhd i$
  \item $i \leq j$
  \end{enumerate}
\end{definition}

\begin{remark}
  In the case when the two pomonoids share a common unit the last
  three conditions for a duoidal relationship are automatically
  satisfied.
\end{remark}

\begin{remark}
  If $\bullet$ is a join, or $\rhd$ is a meet, then these conditions
  are automatically met.
\end{remark}

\subsection{MAV-algebras}

\begin{definition}
  An \emph{MAV-algebra} is a structure
  $(A, \leq, \otimes, ;, I, \lnot)$ with the following properties:
  \begin{enumerate}
  \item $(A, \leq, \otimes, I, \lnot)$ is $*$-autonomous and satisfies \emph{mix}.
  \item $(A, \leq, ;, I)$ is a pomonoid.
  \item $(\otimes, I)$ and $(;, I)$ are in a duoidal relationship.
  \item $(A, \leq)$ has binary meets, which we write as $x \with y$.
  \end{enumerate}
\end{definition}

\begin{proposition}\label{prop:mav-algebra-consequences}
  \bob{Consequences of the definition, as listed in the Agda}
\end{proposition}

\begin{definition}\label{defn:mav-interpretation}
  Let $\mathit{At}$ be a set of atomic propositions. Given an
  MAV-algebra $(A, \leq, \otimes, \rhd, I, \lnot)$ and valuation
  $V : \mathit{At} \to A$, define the interpretation of MAV Formulas
  as follows:
  \begin{mathpar}
    FIXME
  \end{mathpar}
\end{definition}

\begin{theorem}\label{thm:soundness}
  The interpretation defined in Definition
  \ref{defn:mav-interpretation} is sound: for all formulas $P$, if
  $P \longrightarrow^* I$, then $I \leq \llbracket P \rrbracket$.
\end{theorem}

\begin{proof}
  Each of the required inequalities has been established in
  Proposition \ref{prop:mav-algebra-consequences}.
\end{proof}

\subsection{MAV-frames}

To prove completeness of the cut-free fragment of SMAV, we will
construct a particular MAV-algebra from the syntax and cut-free
proofs. Since cut-free proofs do not a priori have all the necessary
structure for an MAV-algebra, we develop a procedure to construct
MAV-algebra from the lighter requirements of an MAV-frame, defined in
this section. MAV-frames have an intuitive reading as a kind of
process algebra, so we use process algebra-like notation. In the
following sections, we show how every MAV-frame induces an
MAV-algebra, and in Section XXX we will show that the MAV-algebra
constructed from the cut-free proof MAV-frame allows us to prove that
all proofs in SMAV have cut-free normal forms.

\begin{definition}
  An \emph{MAV-frame} is a structure
  $(F, \leq, I, \parallel, \rhd, +)$ where $(F, \leq)$ is a partial
  order, $(F, \leq, I, \parallel)$ is a commutative pomonoid,
  $(F, \leq, I, \rhd)$ is a pomonoid, $+$ is a binary monotone
  function on $(F, \leq)$, and these structures satisfy the following
  inequalities:
  \begin{enumerate}
  \item $(w \rhd x) \parallel (y \rhd z) \leq (w \parallel y) \rhd (x \parallel z)$
  \item $(x + y) \parallel z \leq (x \parallel z) + (y \parallel z)$
  \item $(w \rhd x) + (y \rhd z) \leq (w + y) \rhd (x + z)$
  \item $I + I \leq I$
  \end{enumerate}
\end{definition}

\begin{remark}
  \bob{A frame is nearly two duoidal structures, except that we don't
    assume that $+$ is a monoid, nor is there a unit element for
    it. If we assumed that $+$ was a lattice join, then (iii) and (iv)
    would automatically be true. We do not include this because these
    facts are not true for the cut-free MAV system.}
\end{remark}

\begin{remark}
  \bob{Connection to process algebras: we can read the ordering as
    ``reduces to'' and the connectives as parallel combination,
    sequential composition, and choice, respectively. If we add in
    atomic elements with duals and an interaction law, then we get
    something like CCS without restriction, but with sequential
    composition.}
\end{remark}

\begin{proposition}
  \bob{The Cut-free system forms an MAV-frame; fwd ref to next section}
\end{proposition}

\subsection{Constructing MAV-algebras from MAV-frames}

\newcommand{\LowerSet}[1]{\widehat{#1}}
\newcommand{\Day}[1]{\mathop{\widehat{#1}}}


\bob{Some introductory text here: explain the steps in the construction and what each one adds.}

\subsubsection{Lower Sets}

\begin{definition}
  Given a partial order $(A, \leq)$, the set of lower sets
  $\LowerSet{A}$ consists of subsets $F \subseteq A$ that are
  down-closed: $x \in F$ and $y \leq x$ implies $y \in F$. Lower sets
  are ordered by inclusion. Define the embedding
  $\eta : A \to \LowerSet{A}$ as $\eta(x) = \{ y \mid y \leq x \}$.
\end{definition}

\begin{proposition}
  For any $(A, \leq)$, the function $\eta$ is monotone, and
  $(\LowerSet{A}, \subseteq)$ has meets and joins given by
  intersection and union respectively.

  \bob{and $\eta$ preserves any meets that exist}
\end{proposition}

\begin{proposition}\label{prop:day-construction}
  If $(\bullet, i)$ is a pomonoid on $(A, \leq)$, then there is a
  corresponding \emph{Day pomonoid} $(\Day{\bullet}, \Day{i})$ on
  $\LowerSet{A}$ defined as
  $F \Day\bullet G = \{ z \mid z \leq x \bullet y, x \in F, y \in G
  \}$ and $\Day{i} = \eta(i)$. Moreover:
  \begin{enumerate}
  \item $(\Day{\bullet}, \Day{i})$ is left and right residuated: there
    exists a function
    $\Day{\rightblackspoon} : \LowerSet{A} \times \LowerSet{A} \to
    \LowerSet{A}$ such that ... FIXME
  \item If $(\bullet, i)$ is a commutative pomonoid, then so is
    $(\Day{\bullet}, \Day{i})$. In this case, the residuals coincide.
  \end{enumerate}
  \bob{Could also say that $\eta$ preserves the monoids}
\end{proposition}

\begin{remark}
  This is the Day construction \cite{day} restricted to the case of
  partial orders.
\end{remark}

\begin{remark}
  When $(A, \leq)$ is an MAV-frame, by \ref{prop:day-construction} we
  obtain two pomonoids $(\Day{\parallel}, \Day{I})$ and
  $(\Day{\rhd}, \Day{I})$ on $\LowerSet{A}$. Moreover, the next
  proposition states that the duoidal relationship between these
  monoids is preserved by the Day construction:
\end{remark}

\begin{proposition}\label{prop:lower-set-duoidal}
  If $(\bullet, i)$ and $(\rhd, j)$ are in a duoidal relationship then
  so are $(\Day{\bullet}, \Day{i})$ and $(\Day{\rhd}, \Day{j})$.
\end{proposition}

\subsubsection{Closed Lower Sets}

\newcommand{\ClosedLowerSet}[1]{\widehat{#1}^+}
\newcommand{\ClosedDay}[1]{\mathop{\widehat{#1}^+}}
\newcommand{\Chu}{\mathrm{Chu}}
\newcommand{\op}{\mathrm{op}}

The Lower Set adds residuals to the pomonoids of an MAV-frame, so the
Chu construction below will allow us to add duals to interpret all of
the multiplicative subsystem of MAV. To incorporate the additives in
such a way that we will be able to derive the proof normalisation
property, we need to refine lower sets to ones that are closed under
$+$-combinations.

This corresponds to a sheaf construction, but we do not assume that
the ``join'' operation is stable with respect to meets.

The following development has proved amenable to mechanisation.

Formal $+$-contexts are the semantic analogue of Horne's \emph{Killing
  contexts}.

\begin{definition}
  Let $(A, \leq)$ be a partial order with a monotone binary operation
  $+ : A \times A \to A$.  Let $F \in \LowerSet{A}$. A \emph{formal
    $+$-context} over $F$ is a binary tree built from constructors
  $\mathsf{leaf} : F \to \mathrm{ctxt}(F)$ and
  $\mathsf{node} : \mathrm{ctxt}(F) \times \mathrm{ctxt}(F) \to
  \mathrm{ctxt}(F)$. We define the \emph{sum} of a context as
  $\mathit{sum}(\mathsf{leaf}~x) = x$ and
  $\mathit{sum}(\mathsf{node}(c,d)) = \mathit{sum}(c) +
  \mathit{sum}(d)$.
\end{definition}

\begin{remark}
  Conceptually higher level view on contexts: if we view a lower set
  $F$ as a monotone function $F : A^\op \to \Omega$, then we can view
  the context construction as a relative monad on lower sets to
  partial orders.  \bob{and this should extend to presheaves and
    categories}.
\end{remark}

\begin{definition}
  A lower set $F \in \LowerSet{A}$ is \emph{$+$-closed} if for all
  $c \in \mathsf{ctxt}(F)$, $\mathit{sum}(c) \in F$. $+$-closed lower
  sets are ordered by subset inclusion and form a partial order
  $(\ClosedLowerSet{A}, \subseteq)$.
\end{definition}

\begin{proposition}
  There is a monotone function
  $\alpha : \LowerSet{A} \to \ClosedLowerSet{A}$ with the following
  properties. It is a Galois reflection. FIXME

  \begin{displaymath}
    \alpha(F) = \{ x \mid c \in \mathrm{ctxt}(F), x \leq \mathit{sum}(c) \}
  \end{displaymath}
\end{proposition}

\begin{proof}
  Explain the Agda, this relies on the fact that $\mathrm{ctxt}$ is a
  (relative) monad.
\end{proof}

\begin{remark}
  By this proposition, $U \circ \alpha$ is a closure operator on
  $\LowerSet{A}$ \cite{davey-priestley}, and the closed elements are
  those of $\ClosedLowerSet{A}$. The next proposition is standard for
  showing that meets and joins exist on the closed elements for some
  closure operator.
\end{remark}

\begin{proposition}
  $(\ClosedLowerSet{A}, \subseteq)$ has all meets and joins. In the
  binary case, meets are defined by intersection and joins are defined
  by $x \lor y = \alpha (U(x) \cup U(y))$.
\end{proposition}

\begin{remark}
  Note that $\alpha$ does not preserve meets.
\end{remark}

\begin{proposition}\label{prop:closed-monoid-distrib}
  \bob{Monoid construction 1, for monoids that distribute over
    $+$. This also has a residual.}
\end{proposition}

\begin{proof}
  $\alpha$ preserves the monoidal structure, so we can use it to prove
  the monoid laws for the constructed monoid.

  \bob{Sketch why this works: we can multiply together the contexts.}
\end{proof}

\begin{proposition}\label{prop:closed-monoid-duoidal}
  For a pomonoid $(\rhd, j)$ on $(A, \leq)$, if this satisfies
  $(w \rhd x) + (y \rhd z) \leq (w + y) \rhd (x + z)$ then the Day
  construction
  $F \Day{\rhd} G = \{ z \mid z \leq x \rhd y, x \in F, y \in G \}$ on
  lower sets is $+$-closed when $F$ and $G$ are. We write
  $F \ClosedDay{\rhd} G$ to indicate when we mean this construction as
  an operation on $+$-closed lower sets. If $j + j \leq j$, then the
  Day unit $\Day{j} = \eta(j)$ is also closed and we write it as
  $\ClosedDay{j} \in \ClosedLowerSet{A}$. Together
  $(\ClosedDay{\rhd}, \ClosedDay{j})$ form a pomonoid on
  $(\ClosedLowerSet{A}, \subseteq)$.
\end{proposition}

\begin{proof}
  \bob{Explain the Agda here}
\end{proof}

\begin{remark}
  Generalising the situation for the unit $j$ in Proposition
  \ref{prop:closed-monoid-duoidal}, $\eta(x)$ is closed for any $x$
  such that $x + x \leq x$. Note that if $+$ were a join on
  $(A, \leq)$, then this would automatically be satisfied.
\end{remark}

\begin{remark}
  We have used the same decoration $\ClosedDay{\bullet}$ and
  $\ClosedDay{\rhd}$ for two separate constructions of pomonoids on
  $+$-closed lower sets. We will be careful to distinguish which we
  mean: a symmetric operator like $\bullet$ will distribute over $+$
  and so $\ClosedDay{\bullet}$ will be constructed by Proposition
  \ref{prop:closed-monoid-distrib}; and a non-symmetric operator like
  $\rhd$ will be entropic with respect to $+$ and so
  $\ClosedDay{\rhd}$ will be constructed by Proposition
  \ref{prop:closed-monoid-duoidal}.
\end{remark}

\begin{remark}
  If we have two $(A, \leq)$ pomonoids with the same unit, then the
  two constructions will yield the same closed lower set.
  \bob{elaborate a bit...}
\end{remark}

\begin{proposition}
  Let $(\bullet, i)$ and $(\rhd, j)$ be pomonoids on $(A, \leq)$ in a
  duoidal relationship, and assume that $(\bullet, i)$ distributes
  over $+$ (as in Proposition \ref{prop:closed-monoid-distrib}) and
  $(\rhd, j)$ is entropic with respect to $+$ (as in Proposition
  \ref{prop:closed-monoid-duoidal}). Then
  $(\ClosedDay{\bullet}, \ClosedDay{i})$ and
  $(\ClosedDay{\rhd}, \ClosedDay{j})$ are in a duoidal relationship on
  $(\ClosedLowerSet{A}, \subseteq)$.
\end{proposition}

\begin{proof}
  \bob{Explain the Agda here}
\end{proof}

\subsubsection{Chu Construction}

To construct suitable MAV-algebras, we use the poset version of the
Chu construction \cite{barr}. The Chu construction builds
$*$-autonomous categories from symmetric monoidal closed categories
with pullbacks. In the preorder case, the requirement for preorders
simplifies to binary meets. Therefore, for this section, we let
$(A, \leq, \bullet, i, \rightblackspoon)$ be a symmetric monoidal
closed preorder with all binary meets.

\begin{definition}
  Let $k$ be an element of $A$. $\Chu(A, k)$ is the partial order with
  elements pairs $(a^+, a^-)$ such that $a^+ \bullet a^- \leq k$, with
  ordering $(a^+,a^-) \sqsubseteq (b^+, b^-)$ when $a^+ \leq b^+$ and
  $b^- \leq a^-$.
\end{definition}

\begin{proposition}
  $\Chu(A, k)$ is a $*$-autonomous partial order with pomonoid and
  negation structure defined as:
  \begin{mathpar}
    (a^+, a^-) \otimes (b^+, b^-) = (a^+ \bullet b^+, (b^+ \rightblackspoon a^-) \land (a^+ \rightblackspoon b^-))

    I = (i, k)

    (a^+,a^-)^\perp = (a^-, a^+)
  \end{mathpar}
\end{proposition}

\begin{remark}
  If we choose $k = i$, then $\Chu(A, i)$ is a $*$-autonomous partial
  order that satisfies \emph{mix}.
\end{remark}

\begin{proposition}
  If $A$ has binary joins, then $\Chu(A, k)$ has binary meets, given
  by $(a^+,a^-) \with (b^+,b^-) = (a^+ \land b^+, a^- \lor b^-)$.
\end{proposition}

\paragraph{Self-dual duoidal structure}
These two theorems give us enough structure to interpret the
multiplicative and additive constructs of MAV. We need to also
interpret the self-dual $P; Q$ connective. Let us assume that we have
an operation $-\rhd- : A \times A \to A$ that we want to induce
self-dual operator on $\Chu(A, k)$. By the way the Chu construction is
defined, we can make an operator that is self-dual by definition:
\begin{displaymath}
  (a^+, a^-) ; (b^+, b^-) = (a^+ \rhd b^+, a^- \rhd b^-)
\end{displaymath}
for this to work, we need to assume duoidality.

And that the dualising object $k$ is a
satisfies $k \rhd k \leq k$. If $\rhd$ has a unit $j$, then $(j, j)$ is a unit if

\begin{proposition}\label{prop:chu-self-dual}
  Let $(\rhd, j)$ be a pomonoid on $(A, \leq)$ such that
  $k \rhd k \leq k$ and $j \leq k$, then
  $x ; y = (x^+ \rhd y^+, x^- \rhd y^-)$ and $J = (j, j)$ form a
  self-dual pomonoid on $\Chu(A, k)$.
\end{proposition}

\begin{proof}
  \bob{show that that two extra bits are needed.}
\end{proof}

\begin{remark}
  When $k = j$, the two conditions in the proposition are
  automatically satisfied. Moreover, if $k = i = j$, then not only
  does the $*$-autonomous structure satisfy \emph{mix}, but we also
  have $I =J$.
\end{remark}

\begin{lemma}
  \bob{entropy on residuals}
\end{lemma}

\begin{proposition}
  If $(\bullet, i)$ and $(\rhd, j)$ are in a duoidal relationship on
  $(A, \leq)$, and $(\rhd, j)$ satisfies the conditions of Proposition
  \ref{prop:chu-self-dual}, then $(\otimes, I)$ and $(;, J)$ are in a
  duoidal relationship on $\Chu(A, k)$.
\end{proposition}

\begin{proof}
  \bob{Explain the Agda}
\end{proof}

\subsection{Construction of MAV-algebras}

\begin{theorem}
  If $(F, \leq, \parallel, \rhd, I, +)$ is an MAV-frame, then
  $\Chu(\ClosedLowerSet{F}, I)$ has the structure of an MAV-algebra.
\end{theorem}

\begin{proof}
  Combining all the propositions above, in the special case when some
  of the units collapse.
\end{proof}
