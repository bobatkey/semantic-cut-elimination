\section{Semantic Models for MAV}\label{sec:mav-semantics}

To prove the normalisation property for all MAV proofs, we use a
semantic technique that is akin to Okada's phase space method and to
Normalisation by Evaluation (NbE). We construct a semantics of the
whole proof system from the system of normal proofs. This semantics is
constructed in such a way that after interpreting a proof, the
(existence of) a normal form can be extracted (or \emph{read back} or
\emph{reified} in NbE terminology) from the semantic proof.

To our knowledge, the semantics of MAV have not been previously
studied in the literature. The rules are an extension of MALL, so an
\emph{MAV-algebra} will partly be a $*$-autonomous partial order
(\Cref{defn:star-autonomous}) with meets and joins. The additional
structure for $\vSeq$ satisfies the conditions of a duoidal category
\cite[Definition 6.1]{Aguiar_2010} (\Cref{defn:duoidal}). We show that
MAV is sound for MAV-algebras in \Cref{thm:soundness}.

To build MAV-algebras from normal proofs, we define the weaker notion
of an \emph{MAV-frame} (\Cref{defn:mav-frame}). We show
that a combination of certain closed lower sets (\Cref{sec:closed-lower-sets}) and the Chu construction (\Cref{sec:chu}) construct an MAV-algebra from any MAV-frame. Much of
these constructions are well-known, but we have new results on lifting
the Day construction of monoids on closed lower sets and the
preservation of duoidal relationships that are required for MAV.

\subsection{Pomonoidal, $*$-autonomous, and Duoidal Structure on Partial Orders}\label{sec:mav-semantics-preliminaries}

The algebraic semantics of MAV is a collection of interacting monoids
on a partial order. We collect here the basic definitions and kinds of
interaction we will need.

\begin{definition}\label[defn]{defn:pomonoid}
  A \emph{partial order monoid (pomonoid)} $(\bullet, i)$ on a poset
  $(A, \leq)$ comprises a binary operator $\bullet : A \times A \to A$
  that is monotone in both arguments and an element $i \in A$ such
  that the usual monoid laws hold. A \emph{commutative pomonoid} is a
  pomonoid where additionally $x \bullet y = y \bullet x$.
\end{definition}

\begin{definition}\label[defn]{defn:residual}
  A commutative pomonoid $(\bullet, i)$ on a poset $(A, \leq)$ is
  \emph{residuated} if there is a function
  $\rightblackspoon : A \times A \to A$ such that $x \bullet y \leq z$
  iff $x \leq y \rightblackspoon z$.
\end{definition}

Linear logic adds a duality, or negation, to a commutative
pomonoid structure. Semantically, duality with commutativity is
captured in the definition of $*$-autonomous category, due to Barr~\cite{Barr_1979}. For our purposes, we need the partial order
analogue, also called a CL algebra by Troelstra~\cite{Troelstra92:lll}.

\begin{definition}\label[defn]{defn:star-autonomous}
  A \emph{$*$-autonomous partial order} is a structure
  $(A, \leq, \otimes, I, \lnot)$ where $(\otimes, I)$ is a pomonoid on
  $(A, \leq)$ and $\lnot : A^\op \to A$ is an anti-monotone and
  involutive operator on $A$, together satisfying
  $x \otimes y \leq \lnot z$ iff $x \leq \lnot (y \otimes z)$.  A
  $*$-autonomous partial order satisfies \emph{mix} if $\lnot I = I$.
\end{definition}

\begin{remark}
  The structure of a $*$-autonomous partial order has a number of
  immediate consequences, but we leave description of these until
  after the definition of MAV-algebra in \Cref{defn:mav-algebra}.
\end{remark}

BV and MAV extend linear logic by adding a non-commutative pomonoid
structure that interacts with the existing pomonoid via a kind of
interchange law (the \cref{rule:Sequence} rule in the proof
system). We follow \cite[Definition 6.1]{Aguiar_2010} generalising
these to the case of monoids with differing units. Their terminology
is of a category having duoidal structure. We find it useful to
describe one pomonoid as being \emph{duoidal over} another to
emphasise the non-symmetric nature of the relationship, and by analogy
with one monoid distributing over another.

\begin{definition}\label[defn]{defn:duoidal}
  A pomonoid $(\bullet, i)$ is \emph{duoidal over} another pomonoid
  $(\lhd, j)$ on a partial order $(A, \leq)$ if the following
  inequalities hold:
  \begin{enumerate}
    \item $(w \lhd x) \bullet (y \lhd z) \leq (w \bullet y) \lhd (x \bullet z)$
    \item $j \bullet j \leq j$
    \item $i \leq i \lhd i$
    \item $i \leq j$
  \end{enumerate}
\end{definition}

\begin{remark}
  In the case when the two pomonoids share a common unit the last
  three conditions for a duoidal relationship are automatically
  satisfied. We can also ignore the existence of the units and just
  describe two binary operators as being duoidal. If $\bullet$ is a
  join, or $\lhd$ is a meet, then all the conditions for a duoidal
  relationship are automatically met.
\end{remark}

\subsection{MAV-algebras}
\label{sec:mav-algebras}

We define MAV-algebras as the algebraic semantics of MAV. The
definition is a direct translation of the rules of MAV into
order-theoretic language, using the definitions we have seen so far.

\begin{definition}\label[defn]{defn:mav-algebra}
  An \emph{MAV-algebra} is a structure
  $(A, \leq, \otimes, \lhd, I, \lnot)$ with the following properties:
  \begin{enumerate}
    \item $(A, \leq, \otimes, I, \lnot)$ is $*$-autonomous and satisfies \emph{mix}.
    \item $(A, \leq, \lhd, I)$ is a pomonoid.
    \item $\lhd$ is self dual: $\lnot (x \lhd y) = (\lnot x)\lhd (\lnot y)$.
    \item $(\otimes, I)$ is duoidal over $(\lhd, I)$.
    \item $(A, \leq)$ has binary meets, which we write as $x \with y$.
  \end{enumerate}
\end{definition}

\begin{proposition}\label[prop]{prop:mav-algebra-consequences}
  Let $(A, \leq, \otimes, \lhd, I, \lnot)$ be a MAV-algebra.
  \begin{enumerate}
    \item There is another commutative pomonoid structure $(\parr, I)$ on
          $(A, \leq)$, defined as
          $x \parr y = \lnot(\lnot x \otimes \lnot y)$.
    \item $(\otimes, I)$ and $(\parr, I)$ are linearly distributive
          \cite{Cockett_1999}:
          $x \otimes (y \parr z) \leq (x \otimes y) \parr z$.
    \item $(A, \leq)$ has binary joins, given by
          $x \oplus y = \lnot (\lnot x \with \lnot y)$.
    \item $\oplus$ distributes over $\otimes$:
          $x \otimes (y \oplus z) = (x \otimes y) \oplus (x \otimes z)$.
    \item $\with$ distributes over $\parr$:
          $(x \parr z) \with (y \parr z) = (x \with y) \parr z$.
    \item $\lhd$ is duoidal over $\parr$:
          $(w \parr x)\lhd (y \parr z) \leq (w\lhd y) \parr (x\lhd z)$.
    \item $\lhd$ is duoidal over $\with$:
          $(w \with x)\lhd (y \with z) \leq (w \lhd y) \with (x\lhd z)$.
    \item $\oplus$ is duoidal over $\lhd$:
          $(w \lhd x) \oplus (y \lhd z) \leq (x \oplus y) \lhd (x \oplus z)$.
  \end{enumerate}
\end{proposition}

\begin{definition}\label[defn]{defn:mav-interpretation}
  Let $\mathit{At}$ be a set of atomic propositions. Given an
  MAV-algebra $(A, \leq, \otimes, \lhd, I, \lnot)$ and valuation
  $V : \mathit{At} \to A$, define the interpretation of MAV Formulas
  as follows:
  $\sem{\vUnit} = I$,
  $\sem{\va} = V(\va)$,
  $\sem{\vDual \va} = \lnot V(\va)$,
  $\sem{\vP\vTens\vQ} = \sem{\vP} \otimes \sem{\vQ}$,
  $\sem{\vP\vParr\vQ} = \sem{\vP} \parr \sem{\vQ}$,
  $\sem{\vP\vSeq\vQ} = \sem{\vP}\lhd\sem{\vQ}$,
  $\sem{\vP\vWith\vQ} = \sem{\vP} \with \sem{\vQ}$, and
  $\sem{\vP\vPlus\vQ} = \sem{\vP} \oplus \sem{\vQ}$.
\end{definition}

\begin{lemma}\label[lem]{lem:dual-ok}
  For all $\vP$, $\sem{\vDual{\vP}} = \lnot \sem{\vP}$.
\end{lemma}

\begin{theorem}\label[thm]{thm:soundness}
  The interpretation in \Cref{defn:mav-interpretation} is sound:
  for all structures $\vP$,
  if $\vP\vInferFrom*\vUnit$,
  then $I \leq \sem{\vP}$.
\end{theorem}

\begin{proof}
  Each of the required inequalities has been established in
  \Cref{prop:mav-algebra-consequences}.
\end{proof}

\begin{remark}
  More generally, if $P \vInferFrom* Q$ in MAV, then
  $\sem{Q} \leq \sem{P}$ in an MAV-algebra. Note that the ordering is
  reversed! It will be reversed again in the definition of MAV-frame.
\end{remark}

\subsection{MAV-frames}
\label{sec:mav-frames}

To prove completeness of the normal proofs of MAV, we will construct a
particular MAV-algebra from the structures and normal proofs. Since
normal proofs do not \emph{a priori} have all the necessary structure
for an MAV-algebra, in the following sections we develop a procedure
to construct MAV-algebra from the lighter requirements of an
MAV-frame. In \Cref{sec:mav-cut-elimination} we will show that
the MAV-algebra constructed from the normal proof MAV-frame allows us
to prove that all proofs in MAV can be normalised to normal proofs.

\begin{definition}\label[defn]{defn:mav-frame}
  An \emph{MAV-frame} is a structure $(F, \leq, \parr, \lhd, i, +)$
  where $(F, \leq)$ is a partial order, $(F, \leq, \parr, i)$ is a
  commutative pomonoid, $(F, \leq, \lhd, i)$ is a pomonoid, $+$ is a
  binary monotone function on $(F, \leq)$, and these data satisfy the
  following inequalities:
  \begin{enumerate}
    \item $(w \lhd x) \parr (y \lhd z) \leq (w \parr y) \lhd (x \parr z)$
    \item $(x + y) \parr z \leq (x \parr z) + (y \parr z)$
    \item $(w \lhd x) + (y \lhd z) \leq (w + y) \lhd (x + z)$
    \item $i + i \leq i$
  \end{enumerate}
\end{definition}

\begin{remark}
  An MAV-frame is essentially two duoidal relationships and a
  distributivity law.
\end{remark}

\begin{remark}
  MAV-frames have a intuitive reading as CCS-like process algebras
  (see Milner \cite{Milner89:CC} for an introduction to CCS). If we
  assume the existence of a collection of ``action'' elements
  $a \in F$ and their duals $\overline{a} \in F$, satisfying
  $a \parr \overline{a} \leq i$, then we can read the constructs of an
  MAV-frame as parallel composition, sequential composition, and
  choice. The ordering is interpreted as a reduction relation. An
  interesting avenue for future work would be to discover to what
  extent MAV can be thought of as a logic for processes in this
  process algebra.
\end{remark}

\begin{remark}\label[remark]{remark:cka}
  MAV-frames (and MAV-algebras) are also very similar to the
  definition of a \emph{Concurrent Kleene Algebra} (CKA) due to Hoare,
  M{\"o}ller, Struth and Wehrman \cite[Definition
    4.1]{Hoare_2011}. One difference is that we do not assume that $+$
  is a join, nor do we assume the existence of infinitary
  joins. Consequently, we have no analogue of the Kleene Star. Another
  difference is that the duoidal relationship is reversed in
  MAV-frames, indicating that MAV-frames capture evolution of
  processes while MAV-algebras and CKA capture properties of
  processes.
\end{remark}

\begin{proposition}\label[prop]{prop:nmav-frame}
  The \emph{normal proof MAV-frame} \NMAV is the partial order arising
  as the quotient of the preorder formed from the structures of MAV
  and $\vP\leq\vQ$ if there is a normal derivation
  $\vP\vInferFrom*\vQ$, defined as
  $(\vSS, \vInferFrom*, \vParr, \vSeq, \vUnit, \vWith)$, where $\vSS$
  is the set of all MAV structures.  The required (in)equalities
  follow directly from the definition of $\vInferFrom*$ for normal
  proofs.
\end{proposition}

\begin{remark}
  The construction of the MAV-frame \NMAV does not use the $\vTens$
  and $\vPlus$ structure of MAV directly.  This structure is recovered
  by duality from the other connectives by the constructions in the
  rest of this section. This corresponds to the fact that the
  \textsc{Co-X} rules in MAV that we wish to show admissible are the
  ones that mention the $\vTens$ and $\vPlus$ connectives, with the
  exception of \cref{rule:Switch}, which has a special role to play in
  \Cref{prop:embedding-sem} in mediating interaction.
\end{remark}

\subsection{Constructing MAV-algebras from MAV-frames}

We construct MAV-algebras from MAV-frames in a three step process. In
\Cref{sec:lower-sets}, we use lower sets and the Day construction to
add meets, joins and residuals for pomonoids to a partial order. This
construction creates joins freely, so we restrict to $+$-closed lower
sets (\ie, order ideals, but not necessarily over a
$\lor$-semilattice) in \Cref{sec:closed-lower-sets} to turn the $+$
operation in MAV-frames into joins. Restricting to $+$-closed lower
sets separates the Day construction of pomonoids into two separate
cases, depending on how the original pomonoid interacts with
$+$. Finally, we create the $*$-autonomous structure using the Chu
construction in \Cref{sec:chu}. The necessary duoidal structure is
maintained through each construction.

\subsubsection{Lower Sets and Day pomonoids}
\label{sec:lower-sets}

\begin{definition}\label[defn]{defn:lower-set}
  Given a partial order $(A, \leq)$, the set of lower sets
  $\LowerSet{A}$ consists of subsets $F \subseteq A$ that are
  down-closed: $x \in F$ and $y \leq x$ implies $y \in F$. Lower sets
  are ordered by inclusion. Define the embedding
  $\eta : A \to \LowerSet{A}$ as $\eta(x) = \{ y \mid y \leq x \}$.
\end{definition}

\begin{proposition}\label[defn]{defn:lower-set-embed-monotone}
  For any $(A, \leq)$, the function $\eta$ is monotone, and
  $(\LowerSet{A}, \subseteq)$ has meets and joins given by
  intersection and union respectively.
  % \bob{and $\eta$ preserves any meets that exist}
\end{proposition}

\begin{proposition}\label[prop]{prop:day-construction}
  If $(\bullet, i)$ is a pomonoid on $(A, \leq)$, then there is a
  corresponding \emph{Day pomonoid} $(\Day{\bullet}, \Day{i})$ on
  $\LowerSet{A}$ defined as
  $F \Day\bullet G = \{ z \mid z \leq x \bullet y, x \in F, y \in G
    \}$ and $\Day{i} = \eta(i)$. Moreover:
  \begin{enumerate}
    \item If $(\bullet, i)$ is a commutative pomonoid, then so is
          $(\Day{\bullet}, \Day{i})$.
    \item $(\Day{\bullet}, \Day{i})$ has left and right residuals, which
          coincide when it is commutative. We will only be interested in
          residuals for commutative pomonoids, which we write as
          $F \rightblackspoon G$.
    \item The embedding preserves the monoid:
          $\eta(x \bullet y) = \eta(x) \Day\bullet \eta(y)$.
  \end{enumerate}
\end{proposition}

\begin{remark}
  \Cref{prop:day-construction} is the Day monoidal product on functor
  categories \cite{Day_1970} restricted to the case of partial orders
  and lower sets.
\end{remark}

\begin{remark}
  When $(A, \leq)$ is an MAV-frame, \Cref{prop:day-construction} gives us two pomonoids
  $(\Day{\parr}, \Day{I})$ and $(\Day{\lhd}, \Day{I})$ on
  $\LowerSet{A}$. Moreover, the next proposition states that the
  duoidal relationship between these monoids is preserved by the Day
  construction:
\end{remark}

\begin{proposition}\label[prop]{prop:lower-set-duoidal}
  If $(\bullet, i)$ is duoidal over $(\lhd, j)$ then
  $(\Day{\bullet}, \Day{i})$ is duoidal over $(\Day{\lhd}, \Day{j})$.
\end{proposition}

\subsubsection{$+$-closed Lower Sets}
\label{sec:closed-lower-sets}

In the Phase Semantics, structures are interpreted as elements that are
fixed points of a closure operator defined by a double negation with
respect to the monoid on the original frame. This closure operator
generates a partial order of ``facts'' whose meets and joins exactly
correspond to the syntactic ones when the original monoid is derived
from the proofs. As we mentioned in the introduction, the presence of
a self-dual operator in BV means that we cannot use double negation
closure, and we have to proceed more deliberately to preserve
join-like structure in an MAV-frame when building MAV-algebras. We do
this by defining $+$-closed lower sets as those that are closed under
finite $+$-combinations of their elements. This leads to a closure
operator on lower sets that allows us to immediately deduce that
$+$-closed lower sets form a lattice. We also preserve the
Day-pomonoids from lower sets, but in two different ways, depending on
how the original pomonoid interacts with $+$. In \Cref{prop:closed-monoid-distrib} we handle pomonoids that distribute
over $+$, and in \Cref{prop:closed-monoid-duoidal} we
handle pomonoids that are duoidal under $+$. We need these
constructions to lift the $\parr$ and $\lhd$ pomonoids from
MAV-frames to $+$-closed lower sets. Finally in this section, we show
that duoidal structure on lower sets from
\Cref{prop:lower-set-duoidal} is preserved in $+$-closed lower sets.

For this section, we assume that $(A, \leq)$ is a partial order with a
monotone binary operation $+ : A \times A \to A$ (we do not assume
that $+$ is a join or even a pomonoid.)

\begin{definition}\label[defn]{defn:closed-lower-set}
  A lower set $F \in \LowerSet{A}$ is \emph{$+$-closed} if $x \in F$
  and $y \in F$ imply $x + y \in F$. $+$-closed lower sets are ordered
  by subset inclusion and form a partial order
  $(\ClosedLowerSet{A}, \subseteq)$.
\end{definition}

\begin{proposition}
  Let $U : \ClosedLowerSet{A} \to \LowerSet{A}$ be the ``forgetful''
  function that forgets the $+$-closed property. There is a monotone
  function $\alpha : \LowerSet{A} \to \ClosedLowerSet{A}$ such that
  for all $F \in \ClosedLowerSet{A}$, $\alpha(U F) = F$ and for all
  $F \in \LowerSet{A}$, $F \subseteq U (\alpha F)$.
\end{proposition}

\begin{proof}
  To define $\alpha$, we close lower sets under all
  $+$-combinations. To this end, for $F \in \LowerSet{A}$, define
  $\mathrm{ctxt}(F)$, the set of all $+$-combinations of $F$
  inductively built from constructors
  $\mathsf{leaf} : F \to \mathrm{ctxt}(F)$ and
  $\mathsf{node} : \mathrm{ctxt}(F) \times \mathrm{ctxt}(F) \to
    \mathrm{ctxt}(F)$. We define the \emph{sum} of a context as
  $\mathit{sum}(\mathsf{leaf}~x) = x$ and
  $\mathit{sum}(\mathsf{node}(c,d)) = \mathit{sum}(c) +
    \mathit{sum}(d)$. Now define:
  $\alpha(F) = \{ x \mid c \in \mathrm{ctxt}(F), x \leq
    \mathit{sum}(c) \}$. This is $+$-closed, by taking the
  $\mathsf{node}$ combination of contexts. $\alpha \circ U$ is
  idempotent because $\alpha$ does not introduce any elements to lower
  sets that are already closed. For arbitrary lower sets $F$,
  $F \subseteq U(\alpha F)$ by the $\mathsf{leaf}$ constructor.
\end{proof}

\begin{definition}
  Define the embedding $\eta^+ : A \to \ClosedLowerSet{A}$ as
  $\eta^+(x) = \alpha(\eta(x))$.
\end{definition}

\begin{remark}
  By this proposition, $U \circ \alpha$ is a closure operator on
  $\LowerSet{A}$ \cite{Davey_2002}, and the closed elements are
  those of $\ClosedLowerSet{A}$. The next proposition is standard for
  showing that meets and joins exist on the closed elements for some
  closure operator.
\end{remark}

\begin{proposition}
  $(\ClosedLowerSet{A}, \subseteq)$ has all meets and joins. In the
  binary case, meets are defined by intersection and joins are defined
  by $F \lor G = \alpha (U F \cup U G)$.
\end{proposition}

\begin{proposition}\label[prop]{prop:closed-eta-preserve-joins}
  $\eta^+(x + y) \subseteq \eta^+(x) \lor \eta^+(y)$.
\end{proposition}

\begin{remark}
  \Cref{prop:closed-eta-preserve-joins} is the reason for
  requiring $+$-closure. This property will allow us to prove the
  crucial embedding property for all structures in \Cref{sec:mav-cut-elimination}.
\end{remark}

\begin{proposition}\label[prop]{prop:closed-monoid-distrib}
  For a commutative pomonoid $(\bullet, i)$ on $(A, \leq)$ that
  distributes over $+$
  ($(x + y) \bullet z \leq (x \bullet z) + (y \bullet z)$), then
  $F \ClosedDay{\bullet} G = \alpha(U F \Day{\bullet} U G)$ and
  $\ClosedDay{j} = \alpha(\Day{j})$ define a residuated commutative
  pomonoid on $\ClosedLowerSet{A}$. Moreover,
  $\eta^+(x \bullet y) = \eta^+(x) \ClosedDay{\bullet} \eta^+(y)$.
\end{proposition}

\begin{proof}
  Define an operation
  $\bullet^c : \mathrm{ctxt}(F) \times \mathrm{ctxt}(G) \to
    \mathrm{ctxt}(F \Day{\bullet} G)$ that ``multiplies'' two trees,
  such that
  $\mathit{sum}(c) \bullet \mathit{sum}(d) \leq \mathit{sum}(c
    \bullet^c d)$, using the distributivity. This allows us to show that
  $\alpha$ preserves the monoid operation:
  $\alpha F \ClosedDay{\bullet} \alpha F = \alpha (F \Day{\bullet}
    G)$. With this, we can show that the monotonicity, associativity,
  unit, and commutativity properties of $\Day{\bullet}$ transfer over
  to $\ClosedDay{\bullet}$. The definition of the residual from lower
  sets is already $+$-closed, by distributivity.
\end{proof}

\begin{proposition}\label[prop]{prop:closed-monoid-duoidal}
  For a pomonoid $(\lhd, j)$ on $(A, \leq)$, if this satisfies
  $(w \lhd x) + (y \lhd z) \leq (w + y) \lhd (x + z)$ then the Day
  construction
  $F \Day{\lhd} G = \{ z \mid z \leq x \lhd y, x \in F, y \in G \}$ on
  lower sets is $+$-closed when $F$ and $G$ are. We write
  $F \ClosedDay{\lhd} G$ to indicate when we mean this construction as
  an operation on $+$-closed lower sets. If $j + j \leq j$, then the
  Day unit $\Day{j} = \eta(j)$ is also closed and we write it as
  $\ClosedDay{j} \in \ClosedLowerSet{A}$. Together
  $(\ClosedDay{\lhd}, \ClosedDay{j})$ form a pomonoid on
  $(\ClosedLowerSet{A}, \subseteq)$. Moreover,
  $\eta^+(x \lhd y) \leq \eta^+(x) \ClosedDay{\lhd} \eta^+(y)$.
\end{proposition}

\begin{proof}
  Since $+$ is duoidal over $(\lhd, j)$, the Day monoid $\Day{\lhd}$
  is automatically $+$-closed by calculation. The monoid structure
  directly transfers. Similarly, $\eta(j)$ is automatically $+$-closed
  since $j + j \leq j$.
\end{proof}

\begin{remark}
  Generalising the situation for the unit $j$ in \Cref{prop:closed-monoid-duoidal}, $\eta(x)$ is closed for any $x$
  such that $x + x \leq x$. Note that if $+$ were a join on
  $(A, \leq)$, then this would automatically be satisfied.
\end{remark}

\begin{remark}
  We have used the same decoration $\ClosedDay{\bullet}$ and
  $\ClosedDay{\lhd}$ for two separate constructions of pomonoids on
  $+$-closed lower sets. We will be careful to distinguish which we
  mean: in our present application, a symmetric operator like
  $\bullet$ will distribute over $+$ and so $\ClosedDay{\bullet}$ will
  be constructed by \Cref{prop:closed-monoid-distrib}; and
  a non-symmetric operator like $\lhd$ will be duoidal under $+$ and
  so $\ClosedDay{\lhd}$ will be constructed by \Cref{prop:closed-monoid-duoidal}.
\end{remark}

\begin{remark}
  If we have two pomonoids on $(A, \leq)$ that share a unit, then the
  two constructions of units in \Cref{prop:closed-monoid-distrib,prop:closed-monoid-duoidal} will yield the same element of
  $\ClosedLowerSet{A}$.
\end{remark}

\begin{proposition}
  If $(\bullet, i)$ is duoidal over $(\lhd, j)$ on $(A, \leq)$, and
  $(\bullet, i)$ distributes over $+$ (as in \Cref{prop:closed-monoid-distrib}) and $+$ is duoidal over
  $(\lhd, j)$ (as in \Cref{prop:closed-monoid-duoidal}),
  then $(\ClosedDay{\bullet}, \ClosedDay{i})$ is duoidal over
  $(\ClosedDay{\lhd}, \ClosedDay{j})$ on
  $(\ClosedLowerSet{A}, \subseteq)$.
\end{proposition}

\begin{proof}
  The duoidal relationship established in \Cref{prop:lower-set-duoidal} carries over thanks to the properties
  of $\alpha$ and $U$. The fact that
  $\ClosedDay{i} \subseteq \ClosedDay{j}$ relies on the condition
  $j + j \leq j$ to collapse $+$-contexts of $j$s.
\end{proof}

\subsubsection{Chu Construction}
\label{sec:chu}

To construct suitable MAV-algebras, we use the partial order version
of the Chu construction \cite[Appendix by Po-Hsiang
  Chu]{Barr_1979}. The Chu construction builds $*$-autonomous categories
from symmetric monoidal closed categories with pullbacks. In the
partial order case, the requirement for pullbacks simplifies to binary
meets. For this section, we let
$(A, \leq, \bullet, i, \rightblackspoon)$ be a partial order with a
residuated pomonoid structure and all binary meets.

\begin{definition}\label[defn]{defn:chu}
  Let $k$ be an element of $A$. $\Chu(A, k)$ is the partial order with
  elements pairs $(a^+, a^-)$ such that $a^+ \bullet a^- \leq k$, with
  ordering $(a^+,a^-) \sqsubseteq (b^+, b^-)$ when $a^+ \leq b^+$ and
  $b^- \leq a^-$. There is a monotone embedding function
  $\ChuEmbed : A \to \Chu(A, k)$ defined as
  $\ChuEmbed(x) = (x, x \rightblackspoon k)$.
\end{definition}

\begin{proposition}
  $(\Chu(A, k), \sqsubseteq)$ has a $*$-autonomous structure defined
  as:
  \begin{displaymath}
    (a^+, a^-) \otimes (b^+, b^-) = (a^+ \bullet b^+, (b^+ \rightblackspoon a^-) \land (a^+ \rightblackspoon b^-))
    \qquad
    I = (i, k)
    \qquad
    \lnot(a^+,a^-) = (a^-, a^+)
  \end{displaymath}
  Moreover,
  $\ChuEmbed(x \bullet y) = \ChuEmbed(x) \otimes \ChuEmbed(y)$ and
  $\ChuEmbed(i) = I$.
\end{proposition}

\begin{remark}
  If we choose $k = i$, then $(\Chu(A, i), \sqsubseteq)$ has
  $*$-autonomous structure that satisfies \emph{mix}.
\end{remark}

\begin{proposition}\label[prop]{prop:chu-meets}
  If $A$ has binary joins, then $(\Chu(A, k), \sqsubseteq)$ has
  binary meets, given by
  $(a^+,a^-) \with (b^+,b^-) = (a^+ \land b^+, a^- \lor b^-)$.
\end{proposition}

\begin{remark}
  Since $(\Chu(A, k), \sqsubseteq, \otimes, I, \lnot)$ is a
  $*$-autonomous partial order, then \Cref{prop:chu-meets}
  also means that $\Chu(A, k)$ has all binary joins, with
  $(a^+, a^-) \oplus (b^+, b^-) = (a^+ \lor b^+, a^- \land b^-)$.
\end{remark}

We now turn to the self-dual duoidal structure required to interpret
the $\vSeq$ connective. First we transfer pomonoids from $(A, \leq)$
to self-dual pomonoids on $(\Chu(A, k), \sqsubseteq)$ provided they
interact well with $k$:
\begin{proposition}\label[prop]{prop:chu-self-dual}
  Let $(\lhd, j)$ be a pomonoid on $(A, \leq)$ such that
  $(\bullet, i)$ is duoidal over $k \lhd k \leq k$ and $j \leq k$,
  then $x \lhd y = (x^+ \lhd y^+, x^- \lhd y^-)$ and $J = (j, j)$ form
  a self-dual pomonoid on $\Chu(A, k)$.
\end{proposition}

\begin{proof}
  $x\lhd y$ is well-defined because
  $(x^+ \lhd y^+) \bullet (x^- \lhd y^-) \leq (x^+ \bullet x^-) \lhd
    (y^+ \bullet y^-) \leq k \lhd k \leq k$. $J$ is well defined because
  $j \bullet j \leq j \leq k$. The pomonoid laws all transfer directly.
\end{proof}

\begin{remark}
  When $k = j$, the two conditions in the proposition are
  automatically satisfied. Moreover, if $k = i = j$, then not only
  does the $*$-autonomous structure satisfy \emph{mix}, but we also
  have $I =J$.
\end{remark}

Finally, we need to show that if $(\bullet, j)$ is duoidal over
$(\lhd, j)$, then their Chu counterparts are in the same
relationship. Due to the use of residuals in the definition of
$\otimes$, we need the following fact about duoidal residuated
pomonoids:

\begin{lemma}\label[lem]{lem:duoidal-residual}
  If $(\bullet, j)$ is duoidal over $(\lhd, j)$ in a partial order
  $(A, \leq)$ and $(\bullet, j)$ has a residual $\rightblackspoon$,
  then
  $(w \rightblackspoon x) \lhd (y \rightblackspoon z) \leq (w \lhd y)
    \rightblackspoon (x \lhd z)$.
\end{lemma}

\begin{remark}
  \Cref{lem:duoidal-residual} is in some sense the
  ``intuitionistic'' version of the duoidal relationship for $\parr$
  arising as the dual of that for $\otimes$ in a $*$-autonomous
  partial order, as we saw in \Cref{prop:mav-algebra-consequences}.
\end{remark}

\begin{proposition}\label[prop]{prop:chu-monoid-duoidal}
  If $(\bullet, i)$ is duoidal over $(\lhd, j)$ on $(A, \leq)$, and
  $(\lhd, j)$ satisfies the conditions of \Cref{prop:chu-self-dual}, then $(\otimes, I)$ and $(\lhd, J)$ are in a
  duoidal relationship on $\Chu(A, k)$.
\end{proposition}

\begin{proof}
  For the positive half of the Chu construction, this is a direct
  consequence of the duoidal relationship. For the negative half, we
  use \Cref{lem:duoidal-residual} and the fact that meets are
  always duoidal.
\end{proof}

\subsubsection{Construction of MAV-algebras from MAV-frames}
\label{sec:algebra-from-frame}

The propositions in the preceeding three sections together prove that
every MAV-frame yields an MAV-algebra:

\begin{theorem}\label[thm]{thm:algebra-from-frame}
  If $(F, \leq, \parr, \lhd, i, +)$ is an MAV-frame, then
  $(\Chu(\ClosedLowerSet{F}, \ClosedDay{i}), \sqsubseteq)$ has the
  structure of an MAV-algebra.
\end{theorem}

With this theorem we can define a notion of validity in MAV in terms
of truth in all MAV-frame generated algebras. By \Cref{thm:soundness},
MAV is sound for this notion of validity:

\begin{theorem}\label[thm]{thm:mavframe-soundness}
  MAV is sound for the MAV-frame semantics: if $P \vInferFrom* \vUnit$
  then for all MAV-frames $F$, $I \sqsubseteq \sem{P}$ in
  $(\Chu(\ClosedLowerSet{F}, \ClosedDay{i}), \sqsubseteq)$.
\end{theorem}
