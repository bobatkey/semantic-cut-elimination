\section{Extensions and Future Work}\label{sec:future-work}
We have presented a semantic proof of generalised Cut elimination for the Multiplicative-Additive System Virtual (MAV), which reduces Horne's approximately 41 page proof to a 7 page proof.

Our proof technique is modular, and can be adapted with relative ease to a variety of related systems.
To evidence this claim, we have adapted our Agda formalisation to prove generalised cut elimination to the following systems:
\begin{description}
  \item[BV]
        The basic system does not have the additives (\ie $\vWith$ and $\vPlus$).
        The proof is a straightforward restriction of our proof for MAV,
        but only relies on lower sets, rather than $+$-closed lower sets.
        See:
        \begin{center}
          \url{\AgdaBaseUrl/BV.CutElim.html}
        \end{center}
  \item[MAUV]
        The multiplicative-additive-unital system adds the additive units (\ie $\top$ and $\mathbf{0}$, using Girard's notation).
        The proof is a straightforward extension of our proof for MAV, and requires the use of lower sets which are $+$-closed as well as $0$-closed.
  \item[NEL]
        Non-commutative exponential logic~\cite{GuglielmiS11} extends BV with the exponentials (\ie $!$ and $?$, using Girard's notation).
\end{description}
The work opens up several paths for future work. The theory developed
here for lifting Day pomonoids to $+$-closed lower sets enables alternative Cut-elimination proofs for other substructural logics, such as MALL and Bunched Implications (Okada's technique has already been applied here by Frumin~\cite{Frumin22:psc}).
We find the technique of using $+$-closed lower sets rather than more opaque closure operators more revealing in how the

We plan to investigate extensions of MAV with exponentials, as in
the System NEL \cite{GuglielmiS11}, and Kleene Star operators which
can be seen as the exponential for the $\vSeq$ connective. Adding a
Kleene Star would tighten the connection with Concurrent Kleene
Algebras we highlighted in \Cref{remark:cka}. It would interesting to
see to what extent MAV can be seen as a logic for processes
represented as elements of MAV-frames. More generally, fixpoint
operators following Baelde \cite{Baelde12} and De, Jafarrahmani and
Saurin \cite{De22:psc}. The latters' use of Okada's technique is not
compatible with Agda's logic because it relies on impredicativity to
construct fixpoints with the double negation closure. We believe that
our more direct predicative technique will be able to use Agda's
inductive and coinductive types.

We also plan to extend our semantics of BV and MAV to a categorical
semantics that considers equalities between proofs as well as
provability. Such a semantics ought to be useful for treating MAV as a
session-types style language, as considered by Ciobanu and Horne
\cite{Ciobanu_2016}. The necessary analogue of MAV-algebras has
already been investigated by Blute, Panangaden and Slavnov
\cite{Blute_2010} as BV-categories, which are Aguiar and Mahajan's
2-monoidal, or duoidal, categories \cite{Aguiar_2010} extended with
duality. The key task will be to categorify the constructions in this
paper to show how the categorical analogue of MAV-frames induces
MAV-categories.
