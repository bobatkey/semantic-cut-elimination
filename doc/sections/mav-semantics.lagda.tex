\section{Semantics}\label{sec:mav-semantics}

\subsection{$*$-autonomous and Duoidal Preorders}

\subsection{MAV Algebras}

\subsection{Interpretation}

\subsection{Chu Construction and Self-Dual Monoids}

\newcommand{\Chu}{\mathrm{Chu}}
\newcommand{\op}{\mathrm{op}}

To construct suitable MAV Algebras, we use the preorder version of the
Chu construction \cite{barr}. The Chu construction builds
$*$-autonomous categories from symmetric monoidal closed categories
with pullbacks. In the preorder case, the requirement for preorders
simplifies to binary meets. Therefore, for this section, we let
$(A, \leq, \bullet, \epsilon, -\kern-9pt\bullet)$ be a symmetric
monoidal closed preorder with all binary meets.

\begin{definition}
  Let $k$ be an element of $A$. $\Chu(A, k)$ is the preorder with
  elements comprised of pairs $(a^+, a^-)$ such that
  $a^+ \bullet a^- \leq k$, with ordering
  $(a^+,a^-) \sqsubseteq (b^+, b^-)$ when $a^+ \leq b^+$ and
  $b^- \leq a^-$.
\end{definition}

\begin{theorem}
  $\Chu(A, k)$ is a $*$-autonomous preorder with symmetric monoidal
  structure and negation defined as:
  \begin{mathpar}
    (a^+, a^-) \otimes (b^+, b^-) = (a^+ \bullet b^+, (b^+ \multimap a^-) \land (a^+ \multimap b^-))

    I = (\epsilon, k)

    (a^+,a^-)^\perp = (a^-, a^+)
  \end{mathpar}
\end{theorem}

\begin{theorem}
  If $A$ has binary joins, then $\Chu(A, k)$ has binary meets, given
  by $(a^+,a^-) \land (b^+,b^-) = (a^+ \land b^+, a^- \lor b^-)$.
\end{theorem}

\paragraph{Self-dual duoidal structure}
These two theorems give us enough structure to interpret the
multiplicative and additive constructs of MAV. We need to also
interpret the self-dual $P; Q$ connective. Let us assume that we have
an operation $-\rhd- : A \times A \to A$ that we want to induce
self-dual operator on $\Chu(A, k)$. By the way the Chu construction is
defined, we can make an operator that is self-dual by definition:
\begin{displaymath}
  (a^+, a^-) ; (b^+, b^-) = (a^+ \rhd b^+, a^- \rhd b^-)
\end{displaymath}
for this to work, we need to assume duoidality.

And that the dualising object $k$ is a
satisfies $k \rhd k \leq k$. If $\rhd$ has a unit $j$, then $(j, j)$ is a unit if


The Chu construction allows us to state
directly that any operator we define on $\Chu(A, k)$ is self dual by
defining as the same

\subsection{Monoids on Lower Sets}

Lower sets: $X : A^\op \to \Omega$.

\begin{proposition}
  FIXME: Day construction.
\end{proposition}

\subsection{Closure Operators on Lower Sets}

need the ``preorder of elements'' $\int X$ of a lower set $X$.

need trees, and their combination.

% what would the category version of this look like? 

\begin{displaymath}
  \mathsf{C}(X) = \{ x \mid t \in \mathrm{Tree}(\int X), x \leq \mathrm{combine}(t) \}
\end{displaymath}

\begin{theorem}
  $\mathsf{C}$ is a closure operator. Moreover, it is lax monoidal wrt
  to one operator, and strong monoidal wrt the other.
\end{theorem}
