\section{Introduction}\label{sec:introduction}

\begin{enumerate}
      \item It makes sense to add a sequencing connective to MALL. This is
            motivated by coherence spaces (Retore), or by thinking in terms of
            session types.
      \item A sequent calculus is not possible (CITE: Tiu, Strassberger),
            but it is possible to define a proof calculus based on deep
            inference. (Also Retore's proof nets, but it turns out they define a
            different logic).
      \item To make sure that a proof system makes sense, we need some kind
            of Cut-elimination theorem that ensures that all proofs can be
            placed into a normal form where it is evident that it is not
            possible to derive all results.
\end{enumerate}

MAV is defined by Horne \cite{Horne15:mav}, who gave a syntactic proof
of cut-elimination and the admissibility of several other rules. We
provide an alternative proof of cut-elimination via a semantic
model. This proof avoids some of the intrincate syntactic reasoning in
Horne's proof and is more modular, as evidenced by the extensions we
present in \autoref{sec:mav-extensions}.

Cut-elimination for MALL by construction of a semantic model is due to
Okada \cite{Okada99:psc}, who shows that the a Phase Space model of
MALL (described by Girard \cite{Girard87:ll} and Troelstra
\cite{TroelstraXX:lnll}) can be constructed from cut-proof proofs. The
completeness of this model directly yields the existence of a cut-free
proof for every proof constructible in the MALL sequent calculus. For
MAV, we cannot use Phase Spaces due to the

The same technique has been used for MALL with fixpoints (CITE:
Saurin) and for Bunched Implications with extensions (CITE: Frumin).
