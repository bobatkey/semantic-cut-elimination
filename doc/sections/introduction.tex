\section{Introduction}\label{sec:introduction}

We present an algebraic semantics and semantic proof of generalised cut-elimination for the multiplicative-additive system MAV~\cite{Horne15:mav}, which extends the basic system BV\footnote{
      BV stands for \underline{B}asic System \underline{V}irtual, owing to an early interpretation of CCS interaction as the pairwise production and annihilation of virtual particles in physics~\cite[\Fn2]{Horne15:mav}.
}~\cite{Guglielmi99:bv,Guglielmi07:sis} with the additives of multiplicative-additive linear logic~\cite[MALL]{Girard87:ll}. The proof technique also extends to the additive units.
Our proofs are constructive and mechanised in Agda~\cite{Agda264}.

MAV and BV are \emph{deep inference} systems. Deep Inference~\cite{Guglielmi14:di} is generalisation of Gentzen's methodology for designing proof systems, which arose from Guglielmi's attempts to relate process algebra~\cite[CCS]{Milner80:CCS,Milner89:CC} to Classical Linear Logic~\cite[CLL]{Girard87:ll}.
The problem with such a relation is that, while the multiplicative connectives of Linear Logic capture parallel composition, no connective of Linear Logic captures \emph{sequential composition}.
Eventually, Guglielmi's attempts yielded BV, which extends Multiplicative Linear Logic~\cite[MLL]{Girard87:ll} with a self-dual non-commutative connective that captures sequential composition.
Such a connective was already present in another extension of Linear Logic, Pomset logic~\cite{Retore97:pomset}, where it arose from the study of coherence space semantics of Linear Logic~\cite[\C8]{GirardTL89:proofs}.
Recently, Nguyễn and Stra{\ss}burger~\cite{NguyenS22:bvisnotpl} showed that, while BV is similar to Pomset logic, the two are not the same, as the theorems of BV form a proper subset of the theorems of Pomset logic.
Neither BV nor Pomset logic has a sequent calculus. Tiu~\cite{Tiu06:sisii} showed that sequent calculus cannot capture BV, and it is assumed that this result extends to Pomset logic.

\emph{Cut Elimination}, or \emph{admissibility of Cut}, is the fundamental property of Gentzen Sequent Calculi systems, which states that proofs using the Cut rule to introduce ``detours'' can be normalised to ones without. Crucial properties such as consistency and decidability follow from Cut-elimination. The Deep Inference analogue of Cut-elimination is the admissibility of the whole \emph{up} fragment of the calculus, which includes the Deep Inference form of Cut (which we describe in \Cref{sec:mav-syntax} below) as well as duals of most of the other rules of the calculus. Admissibility of the up fragment has the same metatheoretic benefits for Deep Inference systems as it does for Sequent Calculus ones.

Guglielmi \cite[\S4.1]{Guglielmi14:di} proves admissibility via the Splitting Theorem, which shows that proofs of conjoined structures can be split into separate subproofs. This is proved by a detailed syntactic analysis of proofs. Horne~\cite{Horne15:mav} gave a syntactic proof of the admissibility of the up fragment for MAV, that extends Guglielmi's technique with further reasoning about the additives. The proof is quite lengthy and involves intrincate syntactic reasoning and the subtle and complex termination measures.

We present an alternative proof of cut-elimination via a semantic model. This proof avoids some of the intrincacy of Horne's proof. We believe our proof to be more robust in the presence of extensions, due to our use of standard constructions such as Day monoids, order ideals, and the Chu construction. We demonsrate this by scaling down to plain BV and up to MAV with additive units.

The technique of demonstrating Cut Elimination by construction of a semantic model for MALL is due to Okada~\cite{Okada99:psc}, who shows that the phase space model of MALL, described by Girard~\cite[\S4.1]{Girard87:ll} and Troelstra~\cite[\C8]{Troelstra92:lll}, can be constructed from cut-free proofs.
The completeness of this model directly yields the existence of a cut-free proof for every proof constructible in the MALL sequent calculus.
The same technique was used by Abrusci~\cite{Abrusci91:psc} for non-commutative linear logic, by De, Jafarrahmani, and Saurin~\cite{De22:psc} for MALL with fixed points, and by Frumin~\cite{Frumin22:psc} for Bunched Implications.

The phase semantic proof of cut-elimination does not easily extend to include the kind of self-dual connective present in BV and its extensions.
The phase space model derives duality by means of double negation with respect to the monoidal structure, which means that any connective has a derived dual. Attempts to extend the model with the non-commutative connective result in two distinct dual non-comutative connectives, reminiscent of the non-commutative tensor and par introduced by Slavnov~\cite{Slavnov19:scmll}.

To our knowledge, all prior work on the metatheory of Deep Inference systems like BV and MAV has been carried out using syntactic techniques such as rewriting with termination measures, or translations into other logics with known Cut-Elimination properties. Our main contribution is the use of semantic techniques to derive the admissibility of identity expansion, cut, and the other \emph{co-} rules of MAV. To this end, we develop a number of results on the semantics of BV and MAV:
\begin{enumerate}
\item In \Cref{sec:mav-algebras}, we propose MAV-algebras as the algebraic counterpart of MAV. In short, an MAV-algebra is a $*$-autonomous partial order with meets, and another partially ordered monoidal structure that is \emph{duoidal} with respect to the $*$-autonomous structure.
\item Normal proofs (our name for the ``up'' fragment) as we define them in the next section do not {\it a priori} support all the structure of an MAV-algebra, so we define the weaker notion of \emph{MAV-frame} (\Cref{defn:mav-frame}). These play the same role as Kripke structures do for modal and substructural logics. The first main technical contribution of the paper is in showing that every MAV-frame can be completed to an MAV-algebra.
\item We then show that MAV-frames are strongly complete (in the terminology of Okada) for MAV by proving that the MAV-frame constructed from normal proofs can be used to deduce that all MAV-provable structures have normal proofs in \Cref{thm:cut-elim}. Completeness in the usual sense also follows in \Cref{thm:completeness}.
\end{enumerate}

We describe the MAV Deep Inference system in \Cref{sec:mav-syntax} and motivate the idea for readers not familiar with such systems.

All of our proofs have been mechanised and checked in the Agda proof assistant \cite{Agda264}. We briefly discuss the mechanisation in \Cref{sec:mechanisation} and provide a hyperlinked guide to the proof relating it to our mathematical development in \Cref{sec:table-of-statments}. Aside from the benefits of checking the proof, the Agda proof is executable and yields a program for actually normalising proofs.

We present our proof in the full generality of MAV, but note that essentially the same proof applies to the subsystem BV as well. We have also extended the proof technique to include the additive units, and to BV with exponentials (System NEL, analysed by Guglielmi and Stra{\ss}burger \cite{Burger_2011,GuglielmiS11}) We discuss further extensions in \Cref{sec:future-work}.
