\section{Introduction}\label{sec:introduction}

We present an algebraic semantics and semantic proof of cut elimination for the multiplicative-additive deep inference calculus MAV~\cite{Horne15:mav}, an extension of multiplicative-additive linear logic~\cite[MALL]{Girard87:ll} with a self-dual non-commutative connective.

Deep Inference~\cite{Guglielmi14:di} is generalisation of Gentzen's methodology for designing proof systems which arose from Guglielmi's attempts to relate process algebras~\cite[CCS]{Milner89:CC,Milner80:CCS} to linear logic~\cite{Girard87:ll}.
The problem is that, while the connectives of linear logic capture parallel composition, no connective captures \emph{sequential composition}.
Eventually, Guglielmi's attempts yielded the basic deep inference calculus, BV~\cite{Guglielmi99:bv,Guglielmi07:sis}, which extends multiplicative linear logic~\cite[MLL]{Girard87:ll} with a self-dual non-commutative connective that does capture sequential composition.
Such a connective was already present in Pomset logic~\cite{Retore97:pomset}, where it arose from the study of coherence spaces~\cite[Chapter 8]{GirardTL89:proofs}, which are a semantics for proofs of linear logic.
While BV is similar to Pomset logic, the two are not the same. Recently, Nguyễn and Stra{\ss}burger~\cite{NguyenS22:bvisnotpl} showed that the theorems of BV are a proper subset of the theorems of Pomset logic.

% Neither BV nor Pomset logic has a sequent calculus.
% ~\cite{Tiu06:sisii}

\begin{itemize}
      \item BV~\cite{Guglielmi07:sis} by Guglielmi is an extension of MLL inspired by process algebras such as CCSand Pomset Logic~\cite{Retore97:pomset}.
      \item MAV by Hornes~\cite{Horne15:mav} extends BV with the additives.
      \item BV adds a non-commutative connective to MLL. Such a connective can be motivated by coherence spaces, or by session types.
      \item A shallow sequent calculus cannot capture BV~\cite{Tiu06:sisii}.
      \item To make sure that a proof system makes sense, we need some kind
            of Cut-elimination theorem that ensures that all proofs can be
            placed into a normal form where it is evident that it is not
            possible to derive all results.
\end{itemize}

MAV is defined by Horne \cite{Horne15:mav}, who gave a syntactic proof
of cut-elimination and the admissibility of several other rules. We
provide an alternative proof of cut-elimination via a semantic
model. This proof avoids some of the intrincate syntactic reasoning in
Horne's proof and is more modular, as evidenced by the extensions we
present in \autoref{sec:mav-extensions}.

Cut-elimination for MALL by construction of a semantic model is due to
Okada \cite{Okada99:psc}, who shows that the a Phase Space model of
MALL (described by Girard \cite{Girard87:ll} and Troelstra
\cite{TroelstraXX:lnll}) can be constructed from cut-proof proofs. The
completeness of this model directly yields the existence of a cut-free
proof for every proof constructible in the MALL sequent calculus. For
MAV, we cannot use Phase Spaces due to the

The same technique has been used for MALL with fixpoints (CITE:
Saurin) and for Bunched Implications with extensions (CITE: Frumin).
