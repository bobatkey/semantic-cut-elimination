\section{Introduction}\label{sec:introduction}

We present an algebraic semantics and semantic proof of cut-elimination for the multiplicative-additive system MAV~\cite{Horne15:mav}, which extends the basic system BV\footnote{
      BV stands for \underline{B}asic System \underline{V}irtual, owing to an early interpretation of CCS interaction as the pair production and annihilation of virtual particles in physics~\cite[\Fn2]{Horne15:mav}.
}~\cite{Guglielmi99:bv,Guglielmi07:sis} with the additives of multiplicative-additive linear logic~\cite[MALL]{Girard87:ll}.

MAV and BV are \emph{deep inference} systems. Deep Inference~\cite{Guglielmi14:di} is generalisation of Gentzen's methodology for designing proof systems, which arose from Guglielmi's attempts to relate process algebra~\cite[CCS]{Milner80:CCS,Milner89:CC} to linear logic~\cite{Girard87:ll}.
The problem with such a relation is that, while the multiplicative connectives of linear logic capture parallel composition, no connective of linear logic captures \emph{sequential composition}.
Eventually, Guglielmi's attempts yielded BV, which extends multiplicative linear logic~\cite[MLL]{Girard87:ll} with a self-dual non-commutative connective that captures sequential composition.
Such a connective was already present in another extension of linear logic, Pomset logic~\cite{Retore97:pomset}, where it arose from the study of coherence space semantics of linear logic~\cite[\C8]{GirardTL89:proofs}.
Recently, Nguyễn and Stra{\ss}burger~\cite{NguyenS22:bvisnotpl} showed that, while BV is similar to Pomset logic, the two are not the same, as the theorems of BV form a proper subset of the theorems of Pomset logic.
Neither BV nor Pomset logic has a sequent calculus. Tiu~\cite{Tiu06:sisii} showed that sequent calculus cannot capture BV, and it is assumed that this result extends to Pomset logic.

Horne~\cite{Horne15:mav} gave a syntactic proof of cut-elimination for MAV. We present an alternative proof of cut-elimination via a semantic model. This proof avoids some of the intrincate syntactic reasoning in Horne's proof and is more modular, as shown by the extensions we present in \cref{sec:mav-extensions}.

The technique of cut-elimination by construction of a semantic model for MALL is due to Okada~\cite{Okada99:psc}, who shows that the phase space model of MALL, described by Girard~\cite[\S4.1]{Girard87:ll} and Troelstra~\cite[\C8]{Troelstra92:lll}, can be constructed from cut-free proofs.
The completeness of this model directly yields the existence of a cut-free proof for every proof constructible in the MALL sequent calculus.
The same technique was used by Abrusci~\cite{Abrusci91:psc} for non-commutative linear logic, by De, Jafarrahmani, and Saurin~\cite{De22:psc} for MALL with fixed points, and by Frumin~\cite{Frumin22:psc} for bunched implications.

The phase semantic proof of cut-elimination does not easily extend to include the kind of self-dual connective present in BV and its extensions.
The phase space model defines duals by means of double negation with respect to the monoidal structure, which means that attempts to extend the model with the self-dual non-commutative connective result in two distinct dual connectives, reminiscent of the non-commutative tensor and par introduced by Slavnov~\cite{Slavnov19:scmll}.
\wen{%
      I think we can't quite say we \emph{cannot} use phase spaces, only that they do not immediately generalise to accommodate self-dual connectives, right?}
\bob{%
      I think we can say ``The Phase Space proof of cut-elimination does not easily extend to the kind of self-dual connective present in BV and its extensions.'' Maybe elaborate by saying that the phase space model defines duality by means of a double negation with respect to one one of the monoidal structures, and ignores the other one completely. So I think it is unlikely that phase spaces can be adapted to BV.}
\wen{%
      Gave it a shot.}